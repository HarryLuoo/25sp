\documentclass[12pt]{article}
\usepackage{fullpage,mathpazo,amsmath,amsfonts,nicefrac,microtype}

\usepackage{hyperref,color,textcomp}
\definecolor{webgreen}{rgb}{0,.35,0}
\definecolor{webbrown}{rgb}{.6,0,0}
\definecolor{RoyalBlue}{rgb}{0,0,0.9}
\hypersetup{
   colorlinks=true, linktocpage=true, pdfstartpage=3, pdfstartview=FitV,
   breaklinks=true, pdfpagemode=UseNone, pageanchor=true, pdfpagemode=UseOutlines,
   plainpages=false, bookmarksnumbered, bookmarksopen=true, bookmarksopenlevel=1,
   hypertexnames=true, pdfhighlight=/O,
   urlcolor=webbrown, linkcolor=RoyalBlue, citecolor=webgreen,
   pdfauthor={Chris H. Rycroft},
   pdfsubject={UW--Madison Math 521 (Spring 2025)},
   pdfkeywords={},
   pdfcreator={pdfLaTeX},
   pdfproducer={LaTeX with hyperref}
}
\hypersetup{pdftitle={Math 521: Sample midterm questions}}

\newcommand{\N}{\mathbb{N}}
\newcommand{\Z}{\mathbb{Z}}
\newcommand{\Q}{\mathbb{Q}}
\newcommand{\R}{\mathbb{R}}
\newcommand{\sep}{\,:\,}
\setlength{\unitlength}{0.5mm}
\renewcommand{\labelitemii}{$\diamond$}

\pagestyle{empty}

\begin{document}
\section*{Math 521: Midterm 1 information}
\begin{itemize}
  \item The course midterm will take place in class on Wednesday March 12,
    from 9:57am--10:42am.
  \item The exam will cover everything in class up to the end of the lecture on
    February 28.
  \item The exam is closed book---no textbooks, notebooks, calculators,
    or smartphones allowed. You will not be expected to know every theorem by
    heart, but you will be expected to remember the basic definitions, such as
    suprema/infima, convergence/divergence, subsequential limits, interiors and
    closures. You should also be familiar with the limit theorems for
    determining convergence of sequences, and the tests for determining series
    convergence and divergence. 
  \item There will be three questions in total. At least one question will
    involve constructing a mathematical proof.
  \item If a question uses the word ``prove'', then you will be expected to
    write down a mathematical proof similar to those in the textbook or given in
    the homework solutions. If a question uses weaker langauge, such as
    ``determine'', ``justify'', or ``compute'', a less rigorous argument will
    still receive full credit.
  \item You can assume a number of basic results, such as
    \begin{itemize}
      \item $\sum n^{-p}$ converges if and only if $p>1$,
      \item the triangle inequality: $|a|+|b|\ge |a+b|$ for all $a,b\in \R$,
      \item $|b|<a$ if and only if $-a<b<a$,
      \item $\lim_{n\to \infty} a^n= 0$ for $|a|<1$,
      \item $\lim_{n\to \infty} n^{1/n} = 1$.
    \end{itemize}
\end{itemize}

\section*{Sample midterm questions}
The following questions are of a similar style to the ones that will be on the
midterm. They are designed to test familiarity with basic concepts, and will
generally be more straightforward than some of the questions on the homework.
\begin{enumerate}
  \item Prove that $1+\sqrt{1+\sqrt{2}}$ is irrational.
  \item Consider the following series defined for $n \in \N$:
    \[
    \sum_n \frac{8^n}{(n!)^2}, \qquad \sum_n \frac{(-1)^n}{\sqrt{n^2+n}}, \qquad \sum_n \frac{6^n}{n^n}, \qquad \sum_n \frac{1}{n+\nicefrac{1}{2}}.
    \]
    For each series, determine whether they converge or diverge. If you make
    use of any of the theorems for determining series properties, you should
    state which one you use.
  \item
    \begin{enumerate}
      \item Let $S$ and $T$ be non-empty bounded subsets of $\R$. Prove that
	$\sup S\cup T = \max \{\sup S,\sup T\}$ and $\sup S \cap T \le \min \{
	\sup S, \sup T\}$.
      \item Extend part (a) to the cases where $S$ and $T$ are not bounded.
      \item Give an example where $\sup S \cap T < \min \{ \sup S, \sup T\}$.
    \end{enumerate}
  \item Suppose that $(s_n)$ is a convergent sequence and $(t_n)$ is a sequence
    that diverges to $\infty$. Prove that
    \[
    \lim_{n \to \infty} s_n+t_n = \infty.
    \]
  \item Consider the two sets
    \[
    A = (0,1] \cup [4,\infty), \qquad B = \left\{\frac{1}{2n} \sep n \in \N\right\}.
    \]
    For each set, determine its maximum and minimum if they exist. For
    each set, determine its supremum and infimum. Detailed proofs are
    not required, but you should justify your answers.
  \item Prove that the functions
    \[
    d_1(x,y) = (x-y)^4, \qquad d_2(x,y) = 1 + |x-y|,
    \]
    \[
    d_3(x,y)=
    \begin{cases}
      x-y & \qquad \textrm{if $x>y$,} \\
      2(y-x) & \qquad \textrm{if $x \le y,$}
    \end{cases}
    \]
    are not metrics on $\R$.
  \item
    \begin{enumerate}
      \item Prove that if $r$ is irrational, then $r^{1/4}$ and $r+1$ are also irrational.
      \item Prove that $(\sqrt{2}+\sqrt{3})^{1/4}+1$ is irrational.
    \end{enumerate}
  \item Let $S$ be a non-empty bounded subset of $\R$. Define $T=\{|x| \sep
    x\in S\}$ to be the set of all absolute values of elements in $S$. Prove
    that $\sup T = \max\{ \sup S, -\inf S\}$.
  \item
    \begin{enumerate}
      \item Prove that $\sqrt{3}-\sqrt{2}$ is irrational.
      \item Consider the sets
        \[
          A = [0,\sqrt{3}-\sqrt{2}] \cap \Q, \qquad B = [0,\sqrt{3}-\sqrt{2}] \cup \Q.
        \]
        For each set, determine its maximum, minimum, supremum, and infimum, or
        state that these values do not exist. Detailed proofs are not required,
        but you should justify your answers.
    \end{enumerate}
  \item Let $(a_n)$ be a sequence and suppose that $\lim_{n\to\infty} n^2a_n=c
    \in \R$. Prove that $\sum a_n$ converges absolutely.
  \item Consider the function $f(x)=4x+x^2$, and the sequence
    $(s_n)$ defined recursively as
    \[
      s_{n+1} = f(s_n).
    \] 
    \begin{enumerate}
      \item If $s_0=1$, prove that $\lim_{n\to\infty} s_n = \infty$.
      \item What is the minimum of $f$ on the interval $[-4,0]$?
      \item Sketch the function $f$ on the interval $[-4,0]$.
      \item If $s_0 \in [-4,0]$, prove that $(s_n)$ has a convergent subsequence.
    \end{enumerate}
  \item
    \begin{enumerate}
      \item Consider the function $d: \R \times \R \to \R$ defined as
        \[
          d(x,y) = \begin{cases}
            0 & \qquad \text{if $x=y$,} \\
            1+|x-y| & \qquad \text{if $x\ne y$.}
          \end{cases}
        \]
        Prove that $d$ is a metric on $\R$.
      \item For the metric space $(\R,d)$, is the set
        \[
          A=\{\tfrac1n \sep n \in \N\} \cup \{0\}
        \]
        compact? Prove your assertion.
    \end{enumerate}
    
\end{enumerate}
\end{document}
