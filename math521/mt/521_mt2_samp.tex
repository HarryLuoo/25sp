\documentclass[12pt]{article}
\usepackage{fullpage,mathpazo,amsmath,amsfonts,nicefrac,microtype}

\usepackage{hyperref,color,textcomp}
\definecolor{webgreen}{rgb}{0,.35,0}
\definecolor{webbrown}{rgb}{.6,0,0}
\definecolor{RoyalBlue}{rgb}{0,0,0.9}
\hypersetup{
   colorlinks=true, linktocpage=true, pdfstartpage=3, pdfstartview=FitV,
   breaklinks=true, pdfpagemode=UseNone, pageanchor=true, pdfpagemode=UseOutlines,
   plainpages=false, bookmarksnumbered, bookmarksopen=true, bookmarksopenlevel=1,
   hypertexnames=true, pdfhighlight=/O,
   urlcolor=webbrown, linkcolor=RoyalBlue, citecolor=webgreen,
   pdfauthor={Chris H. Rycroft},
   pdfsubject={UW--Madison Math 521 (Spring 2025)},
   pdfkeywords={},
   pdfcreator={pdfLaTeX},
   pdfproducer={LaTeX with hyperref}
}
\hypersetup{pdftitle={Math 521: Sample midterm questions}}

\newcommand{\N}{\mathbb{N}}
\newcommand{\Z}{\mathbb{Z}}
\newcommand{\Q}{\mathbb{Q}}
\newcommand{\R}{\mathbb{R}}
\newcommand{\sep}{\,:\,}
\setlength{\unitlength}{0.5mm}
\renewcommand{\labelitemii}{$\diamond$}

\pagestyle{empty}

\begin{document}
\section*{Math 521: Midterm 2 information}
\begin{itemize}
  \item The second course midterm will take place in class on Wednesday, April 9
    from 9:57am--10:42am.
  \item The exam will cover everything in the lectures from March 3--31,
    inclusive.
  \item The exam is closed book -- no textbooks, notebooks, or calculators
    allowed. You will not be expected to know every theorem by heart, but you
    will be expected to remember the basic definitions, such as uniform
    continuity, uniform convergence, a uniformly Cauchy sequence, and the
    radius of convergence for power series. You will also need to remember some
    of the key results, such as the intermediate value theorem.
  \item There will be three questions in total. At least one question will
    involve constructing a mathematical proof.
  \item If a question uses the word ``prove'', then you will be expected to
    write down a mathematical proof similar to those in the textbook or given in
    the homework solutions. If a question uses weaker langauge, such as
    ``determine'', ``justify'', or ``compute'', a less rigorous argument will
    still receive full credit.
  \item You can assume a number of basic results, such as
    \begin{itemize}
      \item $\sum n^{-p}$ converges if and only if $p>1$,
      \item the triangle inequality: $|a|+|b|\ge |a+b|$ for all $a,b\in \R$,
      \item $|b|<a$ if and only if $-a<b<a$,
      \item $\lim_{n\to \infty} a^n= 0$ for $|a|<1$,
      \item $\lim_{n\to \infty} n^{1/n} = 1$.
    \end{itemize}
\end{itemize}

\section*{Sample midterm questions}
The following questions are of a similar style to the ones that will be on the
midterm. They are designed to test familiarity with basic concepts, and will
generally be more straightforward than some of the questions on the homework.
\begin{enumerate}
  \item Let $(f_n)$ be a sequence of uniformly continuous functions on an
    interval $(a,b)$, and suppose that $f_n$ converges uniformly to a function
    $f$. Prove that $f$ is uniformly continous on $(a,b)$.
  \item
    \begin{enumerate}
      \item Find the radius of convergence of the power series
	\[
	f_1(x)=\sum_{n=1}^\infty \frac{x^n}{n^2}, \qquad f_2(x)=\sum_{n=0}^\infty
	\frac{x^{2n}}{2^n}.
        \]
      \item Show that the series
	\[
	f_3(y)=\sum_{n=1}^\infty \frac{1}{n^2} \left( \frac{y}{1+y^2}\right)^n
	\]
	converges for all values of $y\in \R$.
    \end{enumerate}
  \item Let $(f_n)$ be a sequence of continuous functions on $[a,b]$ that
    converges uniformly to $f$ on $[a,b]$. Show that if $(x_n)$ is a sequence
    in $[a,b]$ and if $x_n \to x$, then $\lim_{n\to \infty} f_n(x_n) = f(x)$.
  \item
    \begin{enumerate}
      \item Determine whether the following series converge or diverge:
        \[
          \sum_{n=2}^\infty \frac{1}{\sqrt{n^2-1}}, \qquad \sum_{n=2}^\infty \frac{(-1)^n}{\sqrt{n^2-1}}.
        \]
        If you make use of any of the theorems for determining series
        properties, you should state which ones you use.
      \item By using the results from part (a), or otherwise, determine
        the radius of convergence of the power series
        \[
          \sum_{n=2}^\infty \frac{5^n x^n}{\sqrt{n^2-1}}.
        \]
    \end{enumerate}
  \item Suppose $f$ is continuous on $[0,2]$ and $f(0)=f(2)$. Prove that there exist
    $x$ and $y$ in $[0,2]$ where $|x-y|=1$ and $f(x)=f(y)$. \textit{[Hint: it
    may be useful to consider the function $g(x)=f(x)-f(x+1)$.]}
  \item Let $(f_n)$ be a sequence of functions on $S\subseteq \R$, and suppose
    that $f_n \to f$ uniformly. Prove that $(f_n)$ is uniformly Cauchy on $S$.
  \item
    \begin{enumerate}
      \item Let $(f_n)$ be a sequence of bounded functions of $S\subseteq \R$.
        Suppose that $f_n \to f$ uniformly. Prove that $f$ is a bounded
        function on $S$.
      \item Construct an example set $S \subseteq \R$ and sequence of bounded
        functions $(f_n)$ such that $f_n \to f$ pointwise, but $f$ is not
        bounded.
    \end{enumerate}
\end{enumerate}
\end{document}
