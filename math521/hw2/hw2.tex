% --------------------------------------------------------------
% This part is the preamble, which you don't have to worry about
% To type your solutions, scroll down to where it says "Start here"
% --------------------------------------------------------------

% This defines the formatting for the document
\documentclass[12pt]{article}
\usepackage[left=1in,top=1in,right=1in,bottom=1in]{geometry}
\usepackage{enumitem}
\setlist{noitemsep}
\setlist[enumerate,1]{label=(\alph*)}
\setlist[enumerate,2]{label=(\roman*)}

% This imports the packages which we will need to type math symbols
\usepackage{amsmath,amssymb}

% This defines the "problem" and "solution" environments
\usepackage{amsthm}
\theoremstyle{definition}\newtheorem{problem}{Problem}
\newenvironment{solution}{\begin{proof}[\bfseries\textup{Solution:}]}{\end{proof}}


\newcommand{\N}{\mathbb{N}}
\newcommand{\B}{\mathbb{B}}
\newcommand{\I}{\mathbb{I}}
\newcommand{\Z}{\mathbb{Z}}
\newcommand{\Q}{\mathbb{Q}}
\newcommand{\R}{\mathbb{R}}
\newcommand{\p}{\partial}
\renewcommand{\vec}[1]{\mathbf{#1}}
\newcommand{\vx}{\vec{x}}
\newcommand{\vp}{\vec{p}}
\newcommand{\sep}{\,:\,}
\newcommand{\Trans}{\mathsf{T}}


\begin{document}

% --------------------------------------------------------------
%                         Start here
% --------------------------------------------------------------

% This is the title:
\begin{center}
521 HW2 | Harry Luo 
\end{center}

% Problem 1 -----------------------------------------------------
\begin{problem}[Problem Title or Description]

\end{problem}
\begin{solution}
%



\end{solution}

% Problem 2 -----------------------------------------------------
\newpage
\begin{problem}[Problem Title or Description]
% Type or paste the problem statement here.
\end{problem}
\begin{solution}
% Type your solution here.
\end{solution}

% Continue adding more problems as needed...

\end{document}