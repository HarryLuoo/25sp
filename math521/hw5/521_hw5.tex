\documentclass[12pt]{article}
\usepackage{fullpage,amsmath,amsfonts,mathpazo,microtype,nicefrac}

\usepackage[font=footnotesize,skip=-10pt]{caption}

% Set-up for hypertext references
\usepackage{hyperref,color,textcomp}
\definecolor{webgreen}{rgb}{0,.35,0}
\definecolor{webbrown}{rgb}{.6,0,0}
\definecolor{RoyalBlue}{rgb}{0,0,0.9}
\hypersetup{
   colorlinks=true, linktocpage=true, pdfstartpage=3, pdfstartview=FitV,
   breaklinks=true, pdfpagemode=UseNone, pageanchor=true, pdfpagemode=UseOutlines,
   plainpages=false, bookmarksnumbered, bookmarksopen=true, bookmarksopenlevel=1,
   hypertexnames=true, pdfhighlight=/O,
   urlcolor=webbrown, linkcolor=RoyalBlue, citecolor=webgreen,
   pdfauthor={Chris H. Rycroft},
   pdfsubject={UW--Madison Math 521 (Spring 2025)},
   pdfkeywords={},
   pdfcreator={pdfLaTeX},
   pdfproducer={LaTeX with hyperref}
}
\hypersetup{pdftitle={Math 521: Assignment 5}}

% Macro definitions


\begin{document}
\section*{Math 521: Assignment 5 (due 5~PM, April 4)}
\begin{enumerate}
  \item A real-valued function $f$ on an interval $I$ is said to be
    \textit{Lipschitz continuous} if there exists an $L>0$ such that
    for all $x,y \in I$,
    \begin{equation}
      |f(x)-f(y)| \le L |x-y|.
    \end{equation}
    \begin{enumerate}
      \item Show that if a function is Lipschitz continuous, then it is
        uniformly continuous.
      \item Find an example of a function $g$ defined on an interval $I$
        that is uniformly continuous but not Lipschitz continuous.
    \end{enumerate}

  \item
    \begin{enumerate}
      \item Let $S$ be a subset of $\R$, and let $f:S \to \R$ and $g:\R \to \R$
	be uniformly continuous functions. Prove that the composition $g \circ
	f: S \to \R$ is uniformly continuous.
      \item Let $f$ and $g$ be two uniformly continuous functions from $S$
	to $\R$. Prove that $f+g$ is uniformly continuous.
      \item Show that there exist uniformly continuous functions $f$ and $g$
	from $S$ to $\R$ such that the multiplication $f\cdot g$ is not
	uniformly continuous.
    \end{enumerate}
    
  \item Let $f$ be a uniformly continuous real-valued function on $\R$. Prove
    that there are constants $A$ and $B$ such that $|f(x)|\le A+B|x|$ for all
    $x\in \R$.

  \item
    \begin{enumerate}
      \item Sketch the function $f(x)=(x+1)^{-2} (x-2)^{-1}$.
      \item Determine $\lim_{x\to 2^+} f(x)$, $\lim_{x\to 2^-} f(x)$,
        $\lim_{x\to -1^+} f(x)$, and $\lim_{x \to -1^-} f(x)$.
      \item Determine $\lim_{x\to 2} f(x)$ and $\lim_{x\to-1} f(x)$ if
        they exist.
    \end{enumerate}

  \item Suppose that the limits $L_1 = \lim_{x\to a^+} f_1(x)$ and $L_2 = \lim_{x\to
    a^+} f_2(x)$ exist.
    \begin{enumerate}
      \item Prove that if $f_1(x) \le f_2(x)$ for some interval $(a,b)$, then
        $L_1\le L_2$.
      \item Suppose that $f_1(x) < f_2(x)$ for some interval $(a,b)$. Is it
        always true that $L_1 <L_2$?
    \end{enumerate}

  \item For each of the following power series, find the radius of convergence
    and determine the exact interval of convergence:
    \begin{enumerate}
      \item $\sum_n n^2 x^n$
      \item $\sum_n \left(\frac{x}{n}\right)^n$
      \item $\sum_n x^{n!}$
      \item $\sum_n 5^n x^{2n+1}$
    \end{enumerate}

  \item For $x\in [0,\infty)$, define $f_n(x) = \tfrac{x}{n}$.
    \begin{enumerate}
      \item Find $f(x) = \lim_{n\to\infty} f_n(x)$.
      \item Determine whether $f_n \to f$ uniformly on $[0,1]$.
      \item Determine whether $f_n \to f$ uniformly on $[0,\infty)$.
    \end{enumerate}

  \item
    \begin{enumerate}
      \item Define a sequence of functions on $\R$ as
        \begin{equation}
          f_n(x) = \begin{cases}
            1 & \qquad \text{if $x=1,\tfrac12,\tfrac13,\ldots,\tfrac1n$,} \\
            0 & \qquad \text{otherwise,}
          \end{cases}
        \end{equation}
        and let $f$ be the pointwise limit of $f_n$. Is each $f_n$ continuous at $0$? Does $f_n \to f$ uniformly on $\R$? Is $f$ continuous at 0?

      \item Repeat part (a) for the sequence of functions
        \begin{equation}
          g_n(x) = \begin{cases}
            x & \qquad \text{if $x=1,\tfrac12,\tfrac13,\ldots,\tfrac1n$,} \\
            0 & \qquad \text{otherwise.}
          \end{cases}
        \end{equation}
        
    \end{enumerate}
\end{enumerate}
\end{document}
