% --------------------------------------------------------------
% This part is the preamble, which you don't have to worry about
% To type your solutions, scroll down to where it says "Start here"
% --------------------------------------------------------------

% This defines the formatting for the document
\documentclass[12pt]{article}
\usepackage[left=1in,top=1in,right=1in,bottom=1in]{geometry}
\usepackage{enumitem}
\setlist{noitemsep}
\setlist[enumerate,1]{label=(\alph*)}
\setlist[enumerate,2]{label=(\roman*)}

% This imports the packages which we will need to type math symbols
\usepackage{amsmath,amssymb}

% This defines the "problem" and "solution" environments
\usepackage{amsthm}
\theoremstyle{definition}\newtheorem{problem}{Problem}
\newenvironment{solution}{\begin{proof}[\bfseries\textup{Solution:}]}{\end{proof}}
\newcommand{\N}{\mathbb{N}}
\newcommand{\B}{\mathbb{B}}
\newcommand{\I}{\mathbb{I}}
\newcommand{\Z}{\mathbb{Z}}
\newcommand{\Q}{\mathbb{Q}}
\newcommand{\R}{\mathbb{R}}
\newcommand{\p}{\partial}
\renewcommand{\vec}[1]{\mathbf{#1}}
\newcommand{\vx}{\vec{x}}
\newcommand{\vp}{\vec{p}}
\newcommand{\sep}{\,:\,}
\newcommand{\Trans}{\mathsf{T}}

\begin{document}

% --------------------------------------------------------------
%                         Start here
% --------------------------------------------------------------

% This is the title:
\begin{center}
\bfseries M521 HW1\\ Harry Luo% Fill in the course name and section here
 
\end{center}

% Problem 1 -----------------------------------------------------
\begin{problem}

\end{problem}
\begin{solution}


    
 \textbf{Base Case (n=1):}
       For $n=1$, LHS = $2(1)+1 = 3$. RHS = $3(1)^2 = 3$.
       Since LHS = RHS, the equation holds for $n=1$.
    
 \textbf{Inductive Step:}
       Assume that for some $k \in \mathbb{N}$, $(2k+1) + (2k+3) + \ldots + (4k-1) = 3k^2$ is true.

       We need to prove that the equation holds for $n=k+1$:
       $$(2(k+1)+1) + (2(k+1)+3) + \ldots + (4(k+1)-1) = 3(k+1)^2.$$
       Consider the LHS for $n=k+1$:
       \begin{align*}
       \text{LHS} &= (2k+3) + (2k+5) + \ldots + (4k+3) \\
             &= [(2k+3) + (2k+5) + \ldots + (4k-1)] + (4k+1) + (4k+3) \\
             &= [(2k+1) + (2k+3) + \ldots + (4k-1)] - (2k+1) + (4k+1) + (4k+3) \\
             &= 3k^2 - (2k+1) + (4k+1) + (4k+3) \\
             &= 3k^2 - 2k - 1 + 4k + 1 + 4k + 3 \\
             &= 3k^2 + 6k + 3 \\
             &= 3(k^2 + 2k + 1) \\
             &= 3(k+1)^2 = \text{RHS}
       \end{align*}
       
    
       By mathematical induction, the equation $(2n+1) + (2n+3) + \ldots + (4n-1) = 3n^2$ is true for all $n \in \mathbb{N}$.
\end{solution}

% Problem 2 -----------------------------------------------------
\newpage
\begin{problem} Pell numbers
\end{problem}
\begin{solution}
\subsection*{(a) Prove $ H_{n} $  is true} 

We will prove that $H_n$ is true for all $n \in \N$ using mathematical induction, where $H_n$ is the statement "$P_n = f(n)$ and $P_{n-1} = f(n-1)$", and $f(n) = \frac{(1+\sqrt2)^n - (1-\sqrt2)^n}{2\sqrt2}$.

\subsubsection*{Base Case (n=1):}
   We check if $H_1$ is true, i.e., if $P_1 = f(1)$ and $P_0 = f(0)$.
   $f(1) = \frac{(1+\sqrt2)^1 - (1-\sqrt2)^1}{2\sqrt2} = \frac{2\sqrt2}{2\sqrt2} = 1 = P_1$.
   $f(0) = \frac{(1+\sqrt2)^0 - (1-\sqrt2)^0}{2\sqrt2} = \frac{1 - 1}{2\sqrt2} = 0 = P_0$.
   Thus, $H_1$ is true.

\subsubsection*{Inductive Step:}
   Assume $H_k$ is true for some $k \in \N$. That is, assume $P_k = f(k)$ and $P_{k-1} = f(k-1)$.


   We want to prove $H_{k+1}$, i.e., $P_{k+1} = f(k+1)$ and $P_k = f(k)$. 

   Using the Pell number recurrence relation: $P_{k+1} = 2P_k + P_{k-1}$.

   By the inductive hypothesis, substitute $P_k = f(k)$ and $P_{k-1} = f(k-1)$:
   $$P_{k+1} = 2f(k) + f(k-1) = 2\frac{(1+\sqrt2)^k - (1-\sqrt2)^k}{2\sqrt2} + \frac{(1+\sqrt2)^{k-1} - (1-\sqrt2)^{k-1}}{2\sqrt2}.$$
   Combining terms:
   $$P_{k+1} = \frac{1}{2\sqrt2} \left[ 2(1+\sqrt2)^k - 2(1-\sqrt2)^k + (1+\sqrt2)^{k-1} - (1-\sqrt2)^{k-1} \right].$$
   Factor out $(1+\sqrt2)^{k-1}$ and $(1-\sqrt2)^{k-1}$, and simplify:

   \begin{align*}P_{k+1} &= \frac{1}{2\sqrt2} \left[ (1+\sqrt2)^{k-1}(2(1+\sqrt2) + 1) - (1-\sqrt2)^{k-1}(2(1-\sqrt2) + 1) \right]
  \\ &= \frac{1}{2\sqrt2} \left[ (1+\sqrt2)^{k-1}(1+\sqrt2)^2 - (1-\sqrt2)^{k-1}(1-\sqrt2)^2 \right] \\ &= \frac{1}{2\sqrt2} \left[ (1+\sqrt2)^{k+1} - (1-\sqrt2)^{k+1} \right] \\&= f(k+1).\end{align*}
   Thus, $P_{k+1} = f(k+1)$. Hence, $H_{k+1}$ is true.

    By mathematical induction, $H_n$ is true for all $n \in \N$. Therefore, $P_n = f(n)$ and $P_{n-1} = f(n-1)$ for all $n \in \N$. In particular, $P_n = f(n)$ for all $n \in \N$. Since we verified $P_0 = f(0) = 0$ separately, we conclude $P_n = f(n)$ for all $n \in \N \cup \{0\}$.

\subsection*{(b) Why is $|\lambda - \sqrt{2}|$ small?}

We have $\lambda = \frac{P_8+P_9}{P_9} = 1 + \frac{P_8}{P_9}$.

From part (a), $P_n = \frac{(1+\sqrt2)^n - (1-\sqrt2)^n}{2\sqrt2}$. 

For large $n$, since $|1-\sqrt2| < 1$, $(1-\sqrt2)^n \approx 0$.

So, $P_n \approx \frac{(1+\sqrt2)^n}{2\sqrt2}$ for large $n$.

Then $$\frac{P_8}{P_9} \approx \frac{(1+\sqrt2)^8 / (2\sqrt2)}{(1+\sqrt2)^9 / (2\sqrt2)} = \frac{(1+\sqrt2)^8}{(1+\sqrt2)^9} = \frac{1}{1+\sqrt2}= \frac{\sqrt2-1}{2-1} = \sqrt2-1.$$
Thus, $\frac{P_8}{P_9} \approx \sqrt{2}-1$.
Therefore, $$\lambda = 1 + \frac{P_8}{P_9} \approx 1 + (\sqrt{2}-1) = \sqrt{2}.$$

Hence, $|\lambda - \sqrt{2}|$ is small because $\lambda$ is based on the ratio of consecutive Pell numbers, which for large indices approximates $\sqrt{2}$.  More precisely, as $n \to \infty$, $\frac{P_{n}}{P_{n-1}} \to 1+\sqrt{2}$, so $\frac{P_{n-1}}{P_{n}} \to \frac{1}{1+\sqrt{2}} = \sqrt{2}-1$. For $n=9$, $\frac{P_8}{P_9}$ is already a good approximation of $\sqrt{2}-1$, making $\lambda = 1 + \frac{P_8}{P_9}$ a good approximation of $\sqrt{2}$.
\end{solution}

% Problem 3 -----------------------------------------------------
\newpage
\begin{problem}
    $\sqrt{2} + \sqrt{5}$ is irrational.
\end{problem}
\begin{solution}
 
    \textbf{Proof by Contradiction:}

    Assume, for contradiction, that $\sqrt2+\sqrt5$ is rational.
    Let $r = \sqrt2+\sqrt5$, where $r \in \Q$.
    

    Square both sides of $r = \sqrt2+\sqrt5$:
    \begin{align*} r^2 &= (\sqrt2+\sqrt5)^2 = 7 + 2\sqrt{10} \end{align*}
    Rearrange to isolate $\sqrt{10}$:
    \begin{align*} \sqrt{10} &= \frac{r^2 - 7}{2} \end{align*}
    Since $r \in \Q$, the expression $\frac{r^2 - 7}{2}$ is also rational.
    Thus, if $\sqrt2+\sqrt5$ is rational, then $\sqrt{10}$ must be rational.
    
    We now prove by contradiction that $\sqrt{10}$ is irrational.

    Assume $\sqrt{10}$ is rational, so $\sqrt{10} = \frac{p}{q}$ for integers $p, q$ with $\gcd(p,q) = 1$ and $q \neq 0$.
    Squaring both sides gives $10 = \frac{p^2}{q^2}$, so $p^2 = 10q^2$.
    This means $p^2$ is divisible by $10$, hence divisible by $2$ and $5$. Since $2$ and $5$ are prime, $p$ must be divisible by $2$ and $5$, so $p = 10k$ for some integer $k$.
    Substituting $p=10k$ into $p^2 = 10q^2$:
    \begin{align*} (10k)^2 &= 10q^2 \\ 100k^2 &= 10q^2 \\ q^2 &= 10k^2 \end{align*}
    This means $q^2$ is divisible by $10$, and thus $q$ is divisible by $10$.
    So, both $p$ and $q$ are divisible by $10$, contradicting $\gcd(p,q) = 1$.
    Therefore, $\sqrt{10}$ is irrational.


    \textbf{In conclusion: }
    We have shown that,  if $\sqrt2+\sqrt5$ is rational, then $\sqrt{10}$ is rational.
    However, we then proved that $\sqrt{10}$ is irrational.
    This is a contradiction.
    Therefore, our initial assumption that $\sqrt2+\sqrt5$ is rational is false.

    
    \textbf{Thus proves $\sqrt2+\sqrt5$ is irrational. } 


\end{solution}


%problem 4 -----------------------------------------------------
\newpage
\begin{problem}[Field properties]
\end{problem}

\begin{solution}

    \subsection*{(a) $\mathbb{N}$ (Natural Numbers)}
        
    \textbf{Field Properties:}
    \begin{itemize}
        \item \textbf{Holds:}  
        A1 (Additive Associativity), A2 (Additive Commutativity),  
        M1 (Multiplicative Associativity), M2 (Multiplicative Commutativity),  
        M3 (Multiplicative Identity: $1$), DL (Distributive Law)
        
        \item \textbf{Fails:}  
        A3 (No additive identity $0 \in \mathbb{N}$),  
        A4 (No additive inverses),  
        M4 (No multiplicative inverses except for $1$)
    \end{itemize}
        
    \textbf{Order Properties (O1–O5):}
    \begin{itemize}
        \item \textbf{O1 (Trichotomy):} Holds. For any $a, b \in \mathbb{N}$, exactly one of $a < b$, $a = b$, or $a > b$ is true.
        \item \textbf{O2 (Antisymmetry):} Suppose $a, b \in \mathbb{N}$ satisfy $a \le b$ and $b \le a$. By O1 (trichotomy), if $a \neq b$, then either $a < b$ or $a > b$ would hold, contradicting one of the inequalities. Thus, the only possibility is $a = b$.
        \item \textbf{O3 (Transitivity):} Holds. If $a < b$ and $b < c$, then $a < c$.
        \item \textbf{O4 (Additive Compatibility):} Holds. If $a < b$, then $a + c < b + c$ for all $c \in \mathbb{N}$.
        \item \textbf{O5 (Multiplicative Compatibility):} Holds. If $a < b$ and $c > 0$, then $a \cdot c < b \cdot c$.
    \end{itemize}
        
    $\mathbb{N}$ satisfies all order properties but is missing additive identity/inverses and multiplicative inverses.
        
    \subsection*{(b) $\mathbb{Z}$ (Integers)}
        
    \textbf{Field Properties:}
    \begin{itemize}
        \item \textbf{Holds:}  
        A1, A2, A3 (Additive Identity: $0$), A4 (Additive Inverses),  
        M1, M2, M3 (Multiplicative Identity: $1$), DL
            
        \item \textbf{Fails:}  
        M4 (No multiplicative inverses except for $\pm 1$)
    \end{itemize}
        
    \textbf{Order Properties (O1–O5):}
    \begin{itemize}
        \item \textbf{O1 :} Holds.
        \item \textbf{O2 :} Holds by the same argument as in $\mathbb{N}$: if $a \le b$ and $b \le a$, then trichotomy forces $a = b$.
        \item \textbf{O3 :} Holds.
        \item \textbf{O4 :} Holds.
        \item \textbf{O5 :} Holds.
    \end{itemize}
        
    $\mathbb{Z}$ does not satisfy M4.
    
    \subsection*{(c) $\mathbb{B} = \{0, 1\}$ (Binary Numbers)}
        
    \textbf{Field Properties:}  
    All field axioms hold:
    \begin{itemize}
        \item \textbf{A1 :} For all $a,b,c \in \B$, $(a + b) + c = a + (b + c)$
        \begin{itemize}
            \item $(1 + 1) + 1 = 0 + 1 = 1 = 1 + 0 = 1 + (1 + 1)$
            \item Similar for other combinations.
        \end{itemize}

        \item \textbf{A2 :} For all $a,b \in \B$, $a + b = b + a$
            \begin{itemize}
                \item $0 + 0 = 0 = 0 + 0$
                \item $0 + 1 = 1 = 1 + 0$
                \item $1 + 1 = 0 = 1 + 1$
            \end{itemize}
        
        \item \textbf{A3 :} $0 \in \B$ satisfies $a + 0 = a$ for all $a \in \B$
        \item \textbf{A4 :} For each $a \in \B$, there exists $-a \in \B$ where:
            \begin{itemize}
                \item $-0 = 0$ since $0 + 0 = 0$
                \item $-1 = 1$ since $1 + 1 = 0$
            \end{itemize}
        \item \textbf{M1 :} For all $a,b,c \in \B$, $(a \cdot b) \cdot c = a \cdot (b \cdot c)$
        \item \textbf{M2 :} For all $a,b \in \B$, $a \cdot b = b \cdot a$
            \begin{itemize}
                \item $0 \cdot 1 = 0 = 1 \cdot 0$
                \item $1 \cdot 1 = 1$
            \end{itemize}
        
        \item \textbf{M3 :} $1 \in \B$ satisfies $a \cdot 1 = a$ for all $a \in \B$
        \item \textbf{M4 :} For non-zero elements:
            \begin{itemize}
                \item $1^{-1} = 1$ since $1 \cdot 1 = 1$
            \end{itemize}
        \item \textbf{DL :} For all $a,b,c \in \B$, $a \cdot (b + c) = (a \cdot b) + (a \cdot c)$
    \end{itemize}   
            
    \textbf{Order Properties (O1–O5):}  
    Given $0 \leq 1$:
    \begin{itemize}
        \item \textbf{O1 :} Holds. For $a, b \in \mathbb{B}$, exactly one of the following holds: $a < b$, $a = b$, or $a > b$.
        \item \textbf{O2 :} If $a \le b$ and $b \le a$, then by the trichotomy property the possibilities $a < b$ or $a > b$ are ruled out, leaving only $a = b$.
        \item \textbf{O3 :} Trivially holds (with only two elements, no nontrivial chain exists).
        \item \textbf{O4 :} \textbf{Fails}. For example, $0 < 1$ but $0 + 1 = 1 \not< 1 + 1 = 0$.
        \item \textbf{O5 :} Vacuously holds. The only $c > 0$ is $1$, and $0 \cdot 1 = 0 < 1 \cdot 1 = 1$.
    \end{itemize}
            
    $\mathbb{B}$ is a \textbf{field} but \textbf{not an ordered field} due to the failure of O4.
\end{solution}

%problem 5 -----------------------------------------------------
\newpage
\begin{problem}
\end{problem}
\begin{solution}
\begin{enumerate}
            \item $0 < 1$.
            
            \textbf{Proof:}

                 \textit{Step 1:} By the trichotomy property (O1), exactly one of the following holds:  
                $0 < 1$, $0 = 1$, or $1 < 0$.  
                Since $1 \neq 0$ , we exclude $0 = 1$.
                
                 \textit{Step 2:} Assume for contradiction that $1 < 0$.  
                Add $-1$ to both sides:  
                $$1 + (-1) < 0 + (-1) \implies 0 < -1.$$
                
                 \textit{Step 3:} Multiply $1 < 0$ by $-1$ (which is positive by Step 2):  
                
                By multiplicative compatibility :  
                $1 \cdot (-1) < 0 \cdot (-1) \implies -1 < 0$.  
                But Step 2 gives $0 < -1$, violating antisymmetry (O2).  
                Hence, $1 < 0$ is false.
\\

                \textit{Conclusion:} By trichotomy, $0 < 1$ must hold.
\\      

\item If $0 < a < b$, then $0 < b^{-1} < a^{-1}$ for all $a, b \in \mathbb{R}$.
            
            \textbf{Proof:}

   \textit{Step 1:} Prove $a^{-1} > 0$ and $b^{-1} > 0$:  
                
                Suppose $a^{-1} \leq 0$. Since $a > 0$, multiplying $a^{-1} \leq 0$ by $a$ gives:  
                $a \cdot a^{-1} \leq 0 \implies 1 \leq 0$, contradicting $0 < 1$ (from part (a)).  
                Thus, $a^{-1} > 0$. Similarly, $b^{-1} > 0$.
                
 \textit{Step 2:} Multiply $a < b$ by $a^{-1}b^{-1} > 0$:  
                
                By multiplicative compatibility :  
                $a \cdot (a^{-1}b^{-1}) < b \cdot (a^{-1}b^{-1}) \implies b^{-1} < a^{-1}$.
                
\textit{Step 3:} Combine results:  
                From $b^{-1} > 0$ (Step 1) and $b^{-1} < a^{-1}$ (Step 2), we conclude by transitivity:  
                $$0 < b^{-1} < a^{-1}.$$

\end{enumerate}
\end{solution}

%problem 6 -----------------------------------------------------
\newpage
\begin{problem}
\end{problem}
\begin{solution}

\begin{enumerate}
    \item \textbf{Set} \( A = [1,2) \cup (3,\infty) \)
    \begin{itemize}
        \item \textbf{Minimum:} \( 1 \) (since \( 1 \) is included in the interval \([1,2)\)).
        \item \textbf{Maximum:} Does not exist (the set is unbounded above).
        \item \textbf{Infimum:} \( \inf A = 1 \).
        \item \textbf{Supremum:} \( \sup A = \infty \).
    \end{itemize}

    \item \textbf{Set} \( B = \{ r \in \mathbb{Q} \mid r < 2 \} \)
    \begin{itemize}
        \item \textbf{Minimum:} Does not exist (no smallest rational number less than 2).
        \item \textbf{Maximum:} Does not exist (approaches \( 2 \) but never attains it).
        \item \textbf{Infimum:} \( \inf B = -\infty \).
        \item \textbf{Supremum:} \( \sup B = 2 \).
    \end{itemize}

    \item \textbf{Set} \( C = \{ r \in \mathbb{Q} \mid r^2 < 2 \} \)
    \begin{itemize}
        \item \textbf{Minimum:} Does not exist (approaches \( -\sqrt{2} \) but never attains it in \( \mathbb{Q} \)).
        \item \textbf{Maximum:} Does not exist (approaches \( \sqrt{2} \) but never attains it in \( \mathbb{Q} \)).
        \item \textbf{Infimum:} \( \inf C = -\sqrt{2} \).
        \item \textbf{Supremum:} \( \sup C = \sqrt{2} \).
    \end{itemize}

    \item \textbf{Set} \( D = \left\{ \frac{1}{m} + n \mid m,n \in \mathbb{N} \right\} \)
    \begin{itemize}
        \item \textbf{Minimum:} Does not exist (smallest term approaches \( 1 \) as \( m \to \infty \), but \( 1 \) is not attained).
        \item \textbf{Maximum:} Does not exist (unbounded above as \( n \to \infty \)).
        \item \textbf{Infimum:} \( \inf D = 1 \).
        \item \textbf{Supremum:} \( \sup D = \infty \).
    \end{itemize}

    \item \textbf{Set} \( E = \{ \sqrt{2}, e, \pi \} \)
    \begin{itemize}
        \item \textbf{Minimum:} \( \sqrt{2} \) 
        \item \textbf{Maximum:} \( \pi \) 
        \item \textbf{Infimum:} \( \inf E = \sqrt{2} \).
        \item \textbf{Supremum:} \( \sup E = \pi \).
    \end{itemize}

    \item \textbf{Set} \( F = \{ 2 - x^2 \mid x \in \mathbb{R} \} \)
    \begin{itemize}
        \item \textbf{Minimum:} Does not exist (unbounded below as \( x \to \pm\infty \)).
        \item \textbf{Maximum:} \( 2 \) (attained at \( x = 0 \)).
        \item \textbf{Infimum:} \( \inf F = -\infty \).
        \item \textbf{Supremum:} \( \sup F = 2 \).
    \end{itemize}

\end{enumerate}
\end{solution}

%problem 7 -----------------------------------------------------
\newpage
\begin{problem}
\end{problem}
\begin{solution}
    
\subsection*{(a)}
\textbf{Statement:} $a<b$ if and only if $a<b+\epsilon$ for all $\epsilon>0$.

\textbf{Disprove:} We will show that the statement is false by providing a counterexample.


\subsubsection*{($\not\Longleftarrow$):}
Consider for a counterexample: 
Let $a = 1$ and $b = 1$. Then $a < b$ is false, since $1 \not< 1$.

Since $\epsilon > 0$, we have $1 + \epsilon > 1$, so $1 < 1 + \epsilon$ is true for all $\epsilon > 0$.

Thus, for $a=1$ and $b=1$, the condition $a<b+\epsilon$ for all $\epsilon>0$ is true, but $a<b$ is false.

Therefore, the reverse implication is false, and the statement "$a<b$ if and only if $a<b+\epsilon$ for all $\epsilon>0$" is false.

\textbf{Counterexample:} Let $a=1$ and $b=1$. Then $a < b + \epsilon$ for all $\epsilon > 0$, but $a \not< b$.


\subsection*{(b)}
\textbf{Statement:} $a \le b$ if and only if $a<b+\epsilon$ for all $\epsilon>0$.


\textbf{Proof:}

\subsubsection*{($\implies$)}
Assume $a \le b$.

Since $\epsilon>0$, we know $b < b+\epsilon$.

If $a < b$, then $a < b < b+\epsilon$, so $a < b+\epsilon$.

If $a = b$, then $a = b < b+\epsilon$, so $a < b+\epsilon$.

In both cases, if $a \le b$, then $a < b+\epsilon$ for all $\epsilon>0$. Therefore, the forward direction is true.

\subsubsection*{($\Longleftarrow$)}
Assume $a<b+\epsilon$ for all $\epsilon>0$. Suppose for contradiction that $a > b$.

Let $\delta = a - b$. Since $a > b$, we have $\delta > 0$.

Choose $\epsilon = \frac{\delta}{2} = \frac{a-b}{2}$. Since $\delta > 0$, we have $\epsilon > 0$.

By our assumption, $a < b+\epsilon$ for all $\epsilon>0$, so it must be true for $\epsilon = \epsilon_0 = \frac{a-b}{2}$.

Thus, $$a < b + \epsilon_0 = b + \frac{a-b}{2} = \frac{2b + a - b}{2} = \frac{a+b}{2}$$

So we have $a < \frac{a+b}{2}$. Multiplying both sides by 2 gives $2a < a+b$. Subtracting $a$ from both sides gives $a < b$.

We started by assuming $a > b$ and derived $a < b$, which is a contradiction.

Therefore, we have $a \le b$.

Therefore, the reverse direction is true.

Since both directions are true, the statement "$a \le b$ if and only if $a<b+\epsilon$ for all $\epsilon>0$" is true.
\end{solution}


%problem 8 -----------------------------------------------------
\newpage
\begin{problem}
\end{problem}
\begin{solution}

    
Let \( A \) and \( B \) be non-empty bounded subsets of \( \mathbb{R} \), and define the set
\[
A - B = \{ a - b \mid a \in A, b \in B \}.
\]
We aim to prove that
\[
\sup (A - B) = \sup A - \inf B.
\]

\textbf{Step 1: Prove that \( \sup (A - B) \leq \sup A - \inf B \)}

Let \( x \in A - B \). Then there exist \( a \in A \) and \( b \in B \) such that
\[
x = a - b.
\]
Since \( a \leq \sup A \) (as \( \sup A \) is the least upper bound of \( A \)) and \( b \geq \inf B \) (as \( \inf B \) is the greatest lower bound of \( B \)), we obtain
\[
x = a - b \leq \sup A - \inf B.
\]
Since this holds for all \( x \in A - B \), it follows that
\[
\sup (A - B) \leq \sup A - \inf B.
\]

\textbf{Step 2: Prove that \( \sup (A - B) \geq \sup A - \inf B \)}

We need to show that for any \( \epsilon > 0 \), there exists \( x \in A - B \) such that
\[
x > \sup A - \inf B - \epsilon.
\]

Since \( \sup A \) is the least upper bound of \( A \), there exists \( a' \in A \) such that
\[
a' > \sup A - \frac{\epsilon}{2}.
\]
Similarly, since \( \inf B \) is the greatest lower bound of \( B \), there exists \( b' \in B \) such that
\[
b' < \inf B + \frac{\epsilon}{2}.
\]

Consider \( x = a' - b' \). Then:
\[
\begin{aligned}
x &= a' - b' \\
&> \left( \sup A - \frac{\epsilon}{2} \right) - \left( \inf B + \frac{\epsilon}{2} \right) \\
&= \sup A - \inf B - \epsilon.
\end{aligned}
\]
Thus, for any \( \epsilon > 0 \), there exists \( x \in A - B \) such that \( x > \sup A - \inf B - \epsilon \), which implies
\[
\sup (A - B) \geq \sup A - \inf B.
\]

\textbf{Conclusion:}

Since we have shown both
\[
\sup (A - B) \leq \sup A - \inf B \quad \text{and} \quad \sup (A - B) \geq \sup A - \inf B,
\]
it follows that
\[
\sup (A - B) = \sup A - \inf B.
\]
\end{solution}


%problem 9 -----------------------------------------------------
\newpage
\begin{problem}
\end{problem}
\begin{solution}
\subsubsection*{(a)}
 The statement is \textbf{false}.
    
    \textit{Counterexample:} Let $C = [-2, -1]$ and $D = [3, 4]$.
    
    Then:
    \begin{align*}
        \inf C &= -2 \\
        \inf D &= 3 \\
        (\inf C)(\inf D) &= (-2)(3) = -6
    \end{align*}
    
    However, for $M = \{cd \mid c \in C,\, d \in D\}$:
    \[ \inf M = (-2)(4) = -8 \]
    
    Since $\inf M \neq (\inf C)(\inf D)$, the statement is false.
    

\subsubsection*{(b)}
 The statement is \textbf{true}.
    
    \textit{Proof:} Assume $\sup C < \inf D$. Let $\alpha = \sup C$ and $\beta = \inf D$.
    Define $r = \frac{\alpha + \beta}{2}$. Then:
    \[ \alpha < r < \beta \]
    
    For all $c \in C$: $c \leq \sup C = \alpha < r$, so $c < r$.
    
    For all $d \in D$: $d \geq \inf D = \beta > r$, so $r < d$.
    
    Therefore, for all $c \in C$ and $d \in D$: $c < r < d$.
    

\subsubsection*{(c)}
The statement is \textbf{true}.
    
    \textit{Proof:} Assume there exists $r \in \mathbb{R}$ such that $c < r < d$ for all $c \in C$ and $d \in D$.
    Let $\alpha = \sup C$ and $\beta = \inf D$.
    
    Since $c < r$ for all $c \in C$, $r$ is an upper bound of $C$, so $\alpha \leq r$.
    
    Since $r < d$ for all $d \in D$, $r$ is a lower bound of $D$, so $r \leq \beta$.
    
    Therefore:
    \[ \alpha \leq r \leq \beta \]
    
    Suppose, for contradiction, that $\alpha \geq \beta$. Then:
    \[ \alpha \geq \beta \geq r \geq \alpha \]
    implying $\alpha = \beta = r$.
    
    But then:
    \begin{itemize}
        \item Since $c < r$ for all $c \in C$, no element of $C$ equals $\alpha$
        \item Since $r < d$ for all $d \in D$, no element of $D$ equals $\beta$
    \end{itemize}
    
    By definition of supremum, for any $\epsilon > 0$, there exists $c_\epsilon \in C$ such that:
    \[ \alpha - \epsilon < c_\epsilon < \alpha \]
    
    Similarly, there exists $d_\epsilon \in D$ such that:
    \[ \beta < d_\epsilon < \beta + \epsilon \]
    
    As $\epsilon \to 0$, both sequences approach $r$, making $d_\epsilon - c_\epsilon \to 0$.
    This contradicts the requirement of a non-zero gap between $C$ and $D$ implied by $c < r < d$.
    
    Therefore, $\alpha < \beta$, i.e., $\sup C < \inf D$.
    \end{solution}

%problem 10 -----------------------------------------------------
\newpage
\begin{problem}
\end{problem}

\begin{solution}
    Let $a,b \in \mathbb{R}$ with $a < b$. We will construct an irrational number $x \in \mathbb{I}$ such that $a < x < b$.
    
    Consider the interval $(a-\sqrt{2}, b-\sqrt{2})$. Since $a < b$, we have $a-\sqrt{2} < b-\sqrt{2}$, so this interval is non-empty. 
    
    By the density of rational numbers in $\mathbb{R}$, there exists $r \in \mathbb{Q}$ such that
    \[
    a-\sqrt{2} < r < b-\sqrt{2}
    \]
    
    Let $x = r + \sqrt{2}$. We claim this $x$ satisfies our requirements:
    
    \begin{enumerate}
        \item First, $x \in \mathbb{I}$: Since $r \in \mathbb{Q}$ and $\sqrt{2} \in \mathbb{I}$, their sum $x = r + \sqrt{2}$ must be irrational. (If it were rational, then $\sqrt{2} = x - r$ would be rational, a contradiction.)
        
        \item Second, $a < x < b$: Adding $\sqrt{2}$ to each part of the inequality $a-\sqrt{2} < r < b-\sqrt{2}$ gives:
        \[
        a-\sqrt{2}+\sqrt{2} < r+\sqrt{2} < b-\sqrt{2}+\sqrt{2}
        \]
        which simplifies to
        \[
        a < x < b
        \]
    \end{enumerate}
    
    Therefore, we have constructed an irrational number $x \in \mathbb{I}$ such that $a < x < b$.
\end{solution}

%problem 11 -----------------------------------------------------
\newpage
\begin{problem}
\end{problem}
\begin{solution}
    
(a) For $\displaystyle \frac{n^2+3}{n^2-3}$:
    
Dividing numerator and denominator by $n^2$:
\[ \frac{n^2+3}{n^2-3} = \frac{1 + \frac{3}{n^2}}{1 - \frac{3}{n^2}} \to 1 \text{ as } n \to \infty \]
Therefore, $\lim_{n \to \infty} \frac{n^2+3}{n^2-3} = 1$
\\

(b) For $(-1)^n n$:

When $n$ is even, the term is $n$
When $n$ is odd, the term is $-n$
The sequence alternates between increasingly large positive and negative values.
Therefore, the limit \textbf{does not exist.}
\\

(c) For $\displaystyle \frac{4n+2}{3-5n^2}$:

Dividing numerator and denominator by $n^2$:
\[ \frac{4n+2}{3-5n^2} = \frac{\frac{4}{n} + \frac{2}{n^2}}{\frac{3}{n^2} - 5} \to 0 \text{ as } n \to \infty \]
Therefore, $\lim_{n \to \infty} \frac{4n+2}{3-5n^2} = 0$
\\

(d) For $\sqrt{n}-\sqrt{n-1}$:

Rationalizing the numerator:
\[ \sqrt{n}-\sqrt{n-1} = \frac{1}{\sqrt{n}+\sqrt{n-1}} \to 0 \text{ as } n \to \infty \]
Therefore, $\lim_{n \to \infty} (\sqrt{n}-\sqrt{n-1}) = 0$
\\

(e) For $\sqrt{n^2+n}-n$:

Rewriting as $n(\sqrt{1+\frac{1}{n}}-1)$ and using binomial expansion:
\[ \sqrt{1+\frac{1}{n}} \approx 1 + \frac{1}{2n} - \frac{1}{8n^2} + \cdots \]
Therefore, $\lim_{n \to \infty} (\sqrt{n^2+n}-n) = \frac{1}{2}$
\\

(f) For $\frac{n!}{8^n}$:

Using Stirling's approximation:
\[ \frac{n!}{8^n} \approx \left(\frac{n}{8e}\right)^n\sqrt{2\pi n} \]
Since $\frac{n}{8e} > 1$ for large enough $n$, this grows without bound.
Therefore, $\lim_{n \to \infty} \frac{n!}{8^n} = \infty$
\end{solution}

%problem 12 -----------------------------------------------------
\newpage
\begin{problem}
\end{problem}
\begin{solution}

\subsubsection*{(A)}


We will use the series expansion of the mathematical constant $e$ to construct the sequence.

Recall that the number $e$ is defined by the infinite series:
    \begin{equation}\label{eq:e_definition}
    e = \sum_{k=0}^{\infty} \frac{1}{k!} = 1 + \frac{1}{1!} + \frac{1}{2!} + \frac{1}{3!} + \dots
    \end{equation}
    It is a well-known fact that $e$ is irrational.
 We define the sequence $(q_n)$ as the $n$-th partial sum of the series for $e$:
    \begin{equation}\label{eq:q_n_definition}
    q_n = \sum_{k=0}^{n} \frac{1}{k!} = 1 + \frac{1}{1!} + \frac{1}{2!} + \dots + \frac{1}{n!}
    \end{equation}

\textbf{Rationality of $q_n$:}

 Each term $\dfrac{1}{k!}$ is rational since both the numerator and denominator are integers.

 A finite sum of rational numbers is rational.

 Therefore, each $q_n$ is rational.


\textbf{Limit of $q_n$:}

 By the definition of $e$ in equation \eqref{eq:e_definition}, we have:
        \begin{equation*}
        \lim_{n \to \infty} q_n = \lim_{n \to \infty} \sum_{k=0}^{n} \frac{1}{k!} = e
        \end{equation*}
Since $e$ is irrational, the limit is irrational.

\textbf{In conclusion}, the sequence $\{q_n\}$ consists of rational numbers and converges to the irrational number $e$.

\bigskip

\subsubsection*{(B)}


We will construct a sequence converging to the rational number $1$.

 Define the sequence $(p_n)$ by:
    \begin{equation}\label{eq:p_n_definition}
    p_n = 1 + \frac{\sqrt{2}}{n}, \quad \text{for } n \geq 1
    \end{equation}

    \begin{proof}
    We will show that each $p_n$ is irrational.

    \begin{itemize}
        \item Since $\sqrt{2}$ is irrational and $n$ is a positive integer, $\dfrac{\sqrt{2}}{n}$ is irrational.\footnote{Dividing an irrational number by a non-zero integer yields an irrational number because if $\dfrac{\sqrt{2}}{n}$ were rational, then $\sqrt{2} = n \times \text{(rational)}$ would be rational, which contradicts the irrationality of $\sqrt{2}$.}

        \item Suppose, for contradiction, that $p_n$ is rational for some $n$.

        \item Then $p_n - 1 = \dfrac{\sqrt{2}}{n}$ would be rational (since the difference of two rationals is rational).

        \item This implies that $\sqrt{2} = n(p_n - 1)$ is rational (product of an integer and a rational).

        \item This contradicts the fact that $\sqrt{2}$ is irrational.

        \item Therefore, our assumption is false, and $p_n$ must be irrational for all $n$.
    \end{itemize}
    \end{proof}

    \item \textbf{Limit of $p_n$:}
    \begin{align*}
    \lim_{n \to \infty} p_n &= \lim_{n \to \infty} \left(1 + \frac{\sqrt{2}}{n}\right) \\
    &= 1 + \lim_{n \to \infty} \frac{\sqrt{2}}{n} \\
    &= 1 + 0 \\
    &= 1
    \end{align*}
    The limit is the rational number $1$.


\textbf{Conclusion: }
The sequence $\{p_n\}$ consists of irrational numbers and converges to the rational number $1$.


\end{solution}

\end{document}