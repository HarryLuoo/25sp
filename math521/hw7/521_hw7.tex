\documentclass[12pt]{article}
\usepackage{fullpage,amsmath,amsfonts,mathpazo,microtype,nicefrac}

\usepackage[font=footnotesize,skip=-10pt]{caption}

% Set-up for hypertext references
\usepackage{hyperref,color,textcomp}
\definecolor{webgreen}{rgb}{0,.35,0}
\definecolor{webbrown}{rgb}{.6,0,0}
\definecolor{RoyalBlue}{rgb}{0,0,0.9}
\hypersetup{
   colorlinks=true, linktocpage=true, pdfstartpage=3, pdfstartview=FitV,
   breaklinks=true, pdfpagemode=UseNone, pageanchor=true, pdfpagemode=UseOutlines,
   plainpages=false, bookmarksnumbered, bookmarksopen=true, bookmarksopenlevel=1,
   hypertexnames=true, pdfhighlight=/O,
   urlcolor=webbrown, linkcolor=RoyalBlue, citecolor=webgreen,
   pdfauthor={Chris H. Rycroft},
   pdfsubject={UW--Madison Math 521 (Spring 2025)},
   pdfkeywords={},
   pdfcreator={pdfLaTeX},
   pdfproducer={LaTeX with hyperref}
}
\hypersetup{pdftitle={Math 521: Assignment 7}}

% Macro definitions
\newcommand{\N}{\mathbb{N}}
\newcommand{\B}{\mathbb{B}}
\newcommand{\I}{\mathbb{I}}
\newcommand{\Z}{\mathbb{Z}}
\newcommand{\Q}{\mathbb{Q}}
\newcommand{\R}{\mathbb{R}}
\newcommand{\p}{\partial}
\renewcommand{\vec}[1]{\mathbf{#1}}
\newcommand{\vu}{\vec{u}}
\newcommand{\vv}{\vec{v}}
\newcommand{\vx}{\vec{x}}
\newcommand{\vy}{\vec{y}}
\newcommand{\vp}{\vec{p}}
\newcommand{\sep}{\,:\,}
\newcommand{\Trans}{\mathsf{T}}

\begin{document}
\section*{Math 521: Assignment 7 (due 5~PM, May 2)}
At this point in the course, being able to sketch functions and determine their
properties is an important skill, which greatly helps in understanding concepts
such as continuity, differentiability, and uniform convergence. The first
section of this homework is devoted to this topic.

For questions 1--3 and 4(a), no detailed proofs are required, although you will
need to provide some discussion in words about what is going on. To begin, I
would like you to try and draw the graphs by hand. There are many ways to do
this, such as looking at the behavior as $x\to \pm \infty$, calculating a few
specific points and drawing a line through them, using calculus, or searching
for zeroes of the function. After this, you can confirm your results using a
plotting program if you wish. There are many free ones available, such as
\textit{Gnuplot} (\url{http://gnuplot.info}) or \textit{Matplotlib}
(\url{https://matplotlib.org}).

\begin{enumerate}
  \item Consider the function
    \[
    f(x) = \left\{
    \begin{array}{ll}
      1-|x-1| & \qquad \textrm{if $0\le x\le 2$} \\
      0 & \qquad \textrm{if $x>2$}
    \end{array}
    \right.
    \]
    defined on the interval $[0,\infty)$\footnote[2]{If you try \textit{Gnuplot}, you
    can define this function by typing \texttt{f(x)=x>2?0:1-abs(x-1)}. It can
    then be plotted with \texttt{plot [0:4] f(x)}.}. Draw $f(x)$.
    \begin{enumerate}
      \item Draw $f(x/2)$, $f(x/3)$, and $f(x/4)$, and explain how the shapes
	of these curves relate to $f(x)$.
      \item Draw $2f(x)$, $f(x+\nicefrac{1}{2})$, $f(x)-\nicefrac{1}{2}$ and
	explain how the shapes of these curves relate to $f(x)$.
      \item Draw $|f(x)-\nicefrac{1}{2}|$. Is this function continuous? Is it
	differentiable everywhere?
      \item Draw $f(x^2)$ and $f(x)^2$.
    \end{enumerate}
  \item Consider the sequence of functions
    \[
    f_n(x) = \frac{nx^2}{1+nx^2}
    \]
    defined on the interval $[0,\infty)$.
    \begin{enumerate}
      \item Begin by considering $f_1(x)$. How does it behave as $x\to \infty$?
	How does it look close to $x=0$? Use these facts to draw $f_1(x)$.
      \item Show that $f_n(x)=f_1(\sqrt{n} x)$. By considering question 1(a),
	use this fact to draw several of the $f_n(x)$.
      \item It can be shown that $f_n$ converges pointwise to a function $f$
	defined on $[0,\infty)$ as
	\[
	f(x) = \left\{
	\begin{array}{ll}
	  0 & \qquad \textrm{if $x=0$} \\
	  1 & \qquad \textrm{if $x>0$.}
	\end{array}
	\right.
	\]
	Draw $f(x)$ and draw a strip of width $\epsilon=\nicefrac{1}{4}$
	around $f(x)$. If $f_n\to f$ uniformly, then there exists an
	$N$ such that $n>N$ implies that $f_n$ lies wholly within this strip.
	Use the graph to explain in words why no such $N$ exists,
	so that $f_n$ does not converge uniformly to $f$.
    \end{enumerate}
  \item Consider the sequence of functions defined on $\R$ as
    \[
    f_n(x) = \left\{
    \begin{array}{ll}
      x^n \sin \frac{1}{x} & \qquad \textrm{if $x\ne 0$} \\
      0 & \qquad \textrm{if $x=0$}
    \end{array}
    \right.
    \]
    \begin{enumerate}
      \item Draw the sequence of functions $f_0(x)$, $f_1(x)$ and $f_2(x)$.
	Which of the functions are continuous at $x=0$? Which of them are
	differentiable at $x=0$?
      \item Consider the functions $f_n$ on the interval
	$[-\nicefrac{1}{2},\nicefrac{1}{2}]$, and define $f(x)=0$. By
	considering a strip of width $\epsilon$ around $f(x)$,
	explain why $f_n$ will converge uniformly to $f$ on this interval.
    \end{enumerate}
  \item Consider the function $g_0(x)=|x|$ on $\R$. For $n\in \N$, define
    $g_n(x)=|g_{n-1}(x)-2^{1-n}|$.
    \begin{enumerate}
      \item Draw $g_0(x)$, $g_1(x)$, $g_2(x)$, and $g_3(x)$.
      \item Prove that the functions $g_n$ converge uniformly to a limit $g$ on
	$\R$.
    \end{enumerate}
  \item Suppose $f$ is a continuous function on $[a,b]$, and $f(x)\ge 0$ for
    all $x\in[a,b]$. Prove that if $\int_a^b f = 0$, then $f(x)=0$ for
    all $x\in[a,b]$.
  \item Consider the function
    \begin{equation}
      f(x)=\begin{cases}
        1 & \qquad \text{if $x=2^{-n}$ for $n\in \N$,} \\
        0 & \qquad \text{otherwise.}
      \end{cases}
    \end{equation}
    Prove that $f$ is integrable on $[0,1]$ and that $\int_0^1 f = 0$.
  \item Construct an example of a function where $f(x)^2$ is integrable on
    $[0,1]$ but $f(x)$ is not.
  \item Construct an example of a sequence of functions $(f_n)$ on $[0,1]$ such
    that $f_n\to 0$ pointwise, but the sequence $s_n = \int_0^1 f_n$
    diverges.
  \item
    \begin{enumerate}
      \item Suppose that $g$ is integrable on $[0,1]$ and continuous at 0. Prove
        that
        \begin{equation}
          \lim_{n\to\infty} \int_0^1 g(x^n) dx = g(0).
          \label{eq:lim}
        \end{equation}
      \item Show that the condition in part (a) that $g$ is continuous is
        necessary for Eq.~\eqref{eq:lim} to be true.
    \end{enumerate}
  \item Suppose that $f$ is continuous on $(a,b)$, where $a$ may be $-\infty$
    and $b$ may be $\infty$. If $\int_a^b |f(x)|dx <\infty$, show that the
    integral $\int_a^b f(x) dx$ exists and is finite.
  \item
    \begin{enumerate}
      \item Suppose that the real-valued function $f$ satisfies $|f(x)|\le M$
        for all $x\in[a,b]$. Show that
        \begin{equation}
          \left|f(x)^2-f(y)^2 \right| \le 2M |f(x)-f(y)|.
        \end{equation}
      \item Prove that if $f$ is integrable on $[a,b]$, then so is $f^2$.
      \item Suppose that $f$ and $g$ are integrable on $[a,b]$. By considering
        $(f+g)^2$, or otherwise, show that $fg$ is also integrable on $[a,b]$.
    \end{enumerate}
  \item
    \begin{enumerate}
      \item For any two numbers $u,v\in \R$, prove that $uv \le (u^2+v^2)/2$.
        Let $f$ and $g$ be two integrable functions on $[a,b]$. Prove that if
        $\int_a^b f^2 = 1$ and $\int_a^b g^2 = 1$ then
        \begin{equation}
          \int_a^b fg \le 1.
        \end{equation}
      \item
        Prove the Schwarz inequality, that for any two integrable functions $f$
        and $g$ on an interval $[a,b]$,
        \begin{equation}
          \left| \int_a^b fg\right| \le \left( \int_a^b f^2
          \right)^{1/2} \left( \int_a^b g^2 \right)^{1/2}.
        \end{equation}
      \item Let $X$ be the set of all continuous functions on the interval
        $[a,b]$. For any $f,g\in X$, define
        \begin{equation}
          d(f,g) = \left(\int_a^b |f-g|^2 \right)^{1/2}.
        \end{equation}
        Prove that $d$ is a metric.
    \end{enumerate}
\end{enumerate}

\newpage
\section*{Further optional exercises}
\begin{enumerate}
  \setcounter{enumi}{12}
  \item \textbf{Optional.} Define the sequence of functions on $[0,1]$ as
    $h_0(x)=|x-\nicefrac{1}{2}|$ and $h_n(x)=|h_{n-1}(x)-3^{-n}|$. Draw several
    of the $h_n$. Prove that $h_n$ converges uniformly to a limit $h$.
    \textit{Difficult:} where is $h$ differentiable?    
  \item \textbf{Optional.} Given a function $f$ on $[a,b]$ define the \textit{total variation}
    to be
    \begin{equation}
      Vf = \sup \left\{ \sum_{k=1}^n \left|f(x_k) - f(x_{k-1})\right| \right\},
    \end{equation}
    where the supremum is taken over all partitions $P=\{a=x_0 < x_1 < \ldots <
    x_n=b\}$ of $[a,b]$.
    \begin{enumerate}
      \item If $f$ is continuously differentiable use the fundamental
        theorem of calculus to show $Vf \le \int_a^b |f'|$.
      \item Use the mean value theorem to show that $Vf \ge \int_a^b |f'|$
        and hence that $Vf=\int_a^b |f'|.$
    \end{enumerate}
  \item \textbf{Optional.}
    \begin{enumerate}
      \item By using simple properties of $\sin x$ and $\cos x$, show how to
        define the function $\tan: (-\frac{\pi}{2},\frac{\pi}{2}) \to \R$.
        Prove that it is differentiable, strictly increasing, and neither
        bounded above nor below.
      \item By using inverse function theorems, define $\tan^{-1}: \R \to
        (-\frac{\pi}{2},\frac{\pi}{2})$ and show that
        \begin{equation}
          (\tan^{-1})'(x) = \frac{1}{1+x^2}.
        \end{equation}
      \item
        Prove that for $|x|<1$,
        \begin{equation}
          \tan^{-1} x = \sum_{n=0}^\infty \frac{(-1)^n x^{2n+1}}{2n+1}.
        \end{equation}
      \item By making use of Abel's theorem, or otherwise, show that
        \begin{equation}
          \frac{\pi}{4} = \sum_{n=0}^\infty \frac{(-1)^n}{2n+1}.
        \end{equation}
      \item Calculate $(5+i)^4(239-i)$ and use it to prove Machin's formula
        \begin{equation}
          \frac{\pi}{4} = 4\tan^{-1} \frac{1}{5} - \tan^{-1} \frac{1}{239}.
        \end{equation}
    \end{enumerate}
\end{enumerate}
\end{document}
