\documentclass[12pt]{article}
\usepackage{fullpage,mathpazo,amsmath,amsfonts,nicefrac,microtype}

\usepackage{hyperref,color,textcomp}
\definecolor{webgreen}{rgb}{0,.35,0}
\definecolor{webbrown}{rgb}{.6,0,0}
\definecolor{RoyalBlue}{rgb}{0,0,0.9}
\hypersetup{
   colorlinks=true, linktocpage=true, pdfstartpage=3, pdfstartview=FitV,
   breaklinks=true, pdfpagemode=UseNone, pageanchor=true, pdfpagemode=UseOutlines,
   plainpages=false, bookmarksnumbered, bookmarksopen=true, bookmarksopenlevel=1,
   hypertexnames=true, pdfhighlight=/O,
   urlcolor=webbrown, linkcolor=RoyalBlue, citecolor=webgreen,
   pdfauthor={Chris H. Rycroft},
   pdfsubject={UW--Madison Math 521 (Spring 2025)},
   pdfkeywords={},
   pdfcreator={pdfLaTeX},
   pdfproducer={LaTeX with hyperref}
}
\hypersetup{pdftitle={Math 521: Sample final questions}}

\newcommand{\N}{\mathbb{N}}
\newcommand{\Z}{\mathbb{Z}}
\newcommand{\Q}{\mathbb{Q}}
\newcommand{\R}{\mathbb{R}}
\newcommand{\p}{\partial}
\newcommand{\sep}{\,:\,}
\newcommand{\tquad}{\,\,\quad}
\setlength{\unitlength}{0.5mm}
\renewcommand{\vec}[1]{\mathbf{#1}}
\newcommand{\iab}{\int_a^b}
\renewcommand{\labelitemii}{$\diamond$}

\begin{document}
\section*{Math 521: final exam information}
\begin{itemize}
  \item The final exam will take place on Wednesday May 7th from 2:45pm--4:45pm
    in Social Sciences 6203.
  \item The exam will cover all parts of the course with equal weighting. It
    will cover Chapters 1--5, 7--15, 17--21, 23--34, 36 of the Ross textbook.
  \item The final will consist of seven questions, which will be of a similar
    style to those on the midterm. This will give you longer per question than
    on the midterm.
  \item The exam will be closed book---no textbooks, notebooks, or smartphones
    allowed. As on the midterms, you will be expected to be familiar with the
    basic definitions, and know the key results, but it will not be necessary to
    remember the proof of every theorem by heart.
\end{itemize}

\section*{Sample final questions}
\begin{enumerate}
  \item Consider the power series
    \[
    \sum_{n=1}^\infty \frac{x^n}{n^3}, \qquad \sum_{n=1}^\infty \frac{x^{3n}}{2n}, \qquad \sum_{n=0}^\infty x^{2n!}.
    \]
    For each power series, determine its radius of convergence $R$. By
    considering the series at $x=\pm R$, determine the exact interval of
    convergence.
  \item
    \begin{enumerate}
      \item Prove by using the definition of convergence only, without using
	limit theorems, that if $(s_n)$ is a sequence converging to $s$, then
	$\lim_{n\to \infty} s_n^2 = s^2$.
      \item Prove by using the definition of continuity, or by using the
	$\epsilon$-$\delta$ property, that $f(x)=x^2$ is a continuous function
	on $\R$.
    \end{enumerate}
  \item Let $f$ be a twice differentiable function defined on the closed
    interval $[0,1]$. Suppose $r,s,t\in [0,1]$ are defined so that $r<s<t$ and
    $f(r)=f(s)=f(t)=0$. Prove that there exists an $x\in (0,1)$ such that
    $f''(x)=0$.
  \item
    \begin{enumerate}
      \item Suppose that $\sum_{n=0}^\infty a_n$ is a convergent series. Define
	a sequence $(b_n)$ according to $b_n=a_{2n}+a_{2n+1}$ for $n\in \N \cup
	\{0\}$. Prove that $\sum_{n=0}^\infty b_n$ converges.
      \item Construct an example of a series $\sum_{n=0}^\infty a_n$ that
	diverges, but that if $(b_n)$ is defined as above, then
	$\sum_{n=0}^\infty b_n$ converges.
    \end{enumerate}
  \item Suppose $f$ a is real-valued continuous function on $\R$ and that
    $f(a)f(b)<0$ for some $a,b\in \R$ where $a<b$. Prove that there exists an
    $x\in (a,b)$ such that $f(x)=0$.
  \item Let $f$ be a real-valued function defined on an interval $[0,b]$ as
    \[
    f(x) = \left\{
    \begin{array}{ll}
      x & \qquad \textrm{for $x\in \Q$,} \\
      0 & \qquad \textrm{for $x\notin \Q$.}
    \end{array}
    \right.
    \]
    Consider a partition $P=\{0=t_0<t_1<\ldots<t_n=b\}$. What are the upper and
    lower Darboux sums $U(f,P)$ and $L(f,P)$? Is $f$ integrable on $[0,b]$?
  \item Let $f$ be a decreasing function defined on $[1,\infty)$, where
    $f(x)\ge 0$ for all $x\in [1,\infty)$. Prove that $\int_1^\infty f(x)dx$
    converges if and only if $\sum_{n=1}^\infty f(n)$ converges. {\it [This
    is essentially a question asking you to prove a general form of the
    integral test.]}
  \item Consider the function defined for $x,y\in \R$ as
    \[
    d(x,y) = \left\{
    \begin{array}{ll}
      1 & \textrm{if $x\ne y$,} \\
      0 & \textrm{if $x=y$.}
    \end{array}
    \right.
    \]
    \begin{enumerate}
      \item Prove that $d$ defines a metric on $\R$.
      \item What is the neighborhood of radius \nicefrac{1}{2} centered on 0?
      \item Consider an arbitrary set $S\subseteq \R$. Is $S$ open? Is $S$ compact?
    \end{enumerate}
  \item Let $\vec{x}=(x_1,x_2)$ and $\vec{y}=(y_1,y_2)$ be in $\R^2$. Consider
    the function
    \[
    d(\vec{x},\vec{y}) = |x_1-y_1| + |x_2-y_2|.
    \]
    \begin{enumerate}
      \item Prove that $d$ is a metric on $\R^2$.
      \item Compute and sketch the neighborhood of radius 1 at $(0,0)$.
    \end{enumerate}
  \item Consider a function $f$ defined on $\R$ which satisfies
    \[
    |f(x)-f(y)|\le(x-y)^2
    \]
    for all $x,y\in \R$. Prove that $f$ is a constant function.
  \item Suppose that $f$ is differentiable on $\R$, and that $2\le f'(x) \le 3$
    for $x\in \R$. If $f(0)=0$, prove that $2x \le f(x) \le 3x$ for all $x\ge
    0$.
  \item Show that if $f$ is integrable on $[a,b]$, then $f$ is integrable on
    every interval $[c,d]\subseteq[a,b]$.
  \item
    \begin{enumerate}
      \item Suppose $r$ is irrational. Prove that $r^{1/3}$ and $r+1$ are
	irrational also.
      \item Prove that $(5+\sqrt{2})^{1/3}+1$ is irrational.
    \end{enumerate}
  \item By using L'H\^opital's rule, or otherwise, evaluate
    \[
    \lim_{x\to 0} \frac{x}{1-e^{-x^2-3x}}, \qquad \lim_{x\to0} \left(\frac{1}{\sin x} - \frac{1}{x}\right),\qquad \lim_{x\to 0} \frac{x^3}{\sin x-x}.
    \]
  \item Let $a\in \R$. Consider the sequence $(s_n)$ defined as
    \[
    s_n = \left\{
    \begin{array}{ll}
      a & \qquad \textrm{if $n$ is odd,} \\
      2^{-n} & \qquad \textrm{if $n$ is even.}
    \end{array}
    \right.
    \]
    Compute $\lim \sup s_n$ and $\lim \inf s_n$. For what value of $a$ does
    $(s_n)$ converge?
  \item Consider the function $f: \R^2 \to \R$ defined as
    \[
    f(x_1,x_2) = \frac{1}{x_1^2+x_2^2+1}.
    \]
    With respect to the usual Euclidean metrics on $\R$ and $\R^2$, prove that
    $f$ is continuous at $(0,0)$ and at $(0,1)$.
  \item
    \begin{enumerate}
      \item Calculate the improper integral
	\[
	\int_0^1 x^{-p} \,dx
	\]
	for the cases when $0<p<1$ and $p>1$.
      \item Prove that
	\[
	\int_0^\infty x^{-p} dx = \infty
	\]
	for all $p\in(0,\infty)$.
    \end{enumerate}
  \item Prove that if $f$ is integrable on $[a,b]$, then
    \[
    \lim_{d\to b^-} \int_a^d f(x)\,dx = \int_a^b f(x)\,dx.
    \]
  \item Let $f(x)=x^2$, and define a sequence $(s_n)$ according
    to $s_1=\lambda$ and $s_{n+1}=f(s_n)$ for $n\in \N$. Prove
    that $(s_n)$ converges for $\lambda \in [-1,1]$, and diverges
    for $|\lambda|>1$.
  \item Consider the three sets
    \[
    A = [0,\sqrt{2}] \cap \Q, \tquad B = \{x^2+x-1\sep x\in \R\}, \tquad C=\{ x\in \R \sep x^2+x-1<0\}.
    \]
    For each set, determine its maximum and minimum if they exist. For
    each set, determine its supremum and infimum. Detailed proofs are
    not required, but you should justify your answers.
  \item Let $f_n(x)=x-x^n$ on $[0,1]$ for $n\in \N$.
    \begin{enumerate}
      \item Prove that $f_n$ converges pointwise to a limit $f$, and determine $f$.
      \item Prove that $f_n$ does not converge uniformly to $f$.
      \item Find an interval $I$ contained in $[0,1]$ on
	which $f_n\to f$ uniformly.
      \item Prove that the $f_n$ are integrable, that $f$ is integrable,
	and that $\int_0^1 f_n \to \int_0^1 f$.
    \end{enumerate}
  \item Define
    \[
    f(x) = \left\{
    \begin{array}{ll}
      1 & \qquad \textrm{if $|x| \le 1$,} \\
      -2 & \qquad \textrm{if $1<|x| \le 2$,} \\
      0 & \qquad \textrm{if $|x|>2$}
    \end{array}
    \right.
    \]
    for $x\in \R$.
    \begin{enumerate}
      \item Calculate $F(x) = \int_0^x f(t) dt$ for $x\in \R$.
      \item Sketch $f$ and $F$.
      \item Compute $F'$ and state the precise range over which $F'$ exists.
	You may make use of the second fundamental theorem of calculus.
    \end{enumerate}
  \item
    \begin{enumerate}
      \item Let $f$ and $g$ be continuous functions on $[a,b]$ such that
	$\iab f=\iab g$. Prove that there exists an $x\in[a,b]$ such that
	$f(x)=g(x)$.
      \item Construct an example of integrable functions $f$ and $g$ on $[a,b]$
	where $\iab f = \iab g$ but that $f(x)\ne g(x)$ for all $x\in [a,b]$.
    \end{enumerate}
  \item Consider the function
    \[
    f(x) = \frac{x}{1+x}.
    \]
    on the interval $[0,\infty)$.
    \begin{enumerate}
      \item Show that $\lim_{x\to \infty} f(x) = 1$, and that $0\le f(x)<1$ for all
	$x \in [0,\infty)$.
      \item Sketch $f$.
      \item Calculate $f'$, $f''$, and use them to construct the partial Taylor series at $x=1$ with the form
	\[
	f_T(x)=\sum_{n=0}^2 \frac{(x-1)^nf^{(n)}(1)}{n!}.
	\]
      \item Show that $f_T$ can be written as a quadratic equation with the
	form $ax^2+bx+c$, and compute $a$, $b$, and $c$.
      \item Add a sketch of $f_T$ to the sketch of $f$. {\it[Note:
	$f_T(1)=f(1)$ so the two curves should intersect at $x=1$.]}
    \end{enumerate}

  \item Determine the radius of convergence $R$ of the power series
    \[
    f_1(x)=\sum_{n=0}^\infty \frac{x^n}{\sqrt{n^2+1}}, \qquad f_2(x)=\sum_{n=1}^\infty \frac{(-2)^n x^{2n}}{n^2}.
    \]
    By considering the series at $x=\pm R$, determine the exact intervals of
    convergence. If you make use of any of the theorems for determining series
    properties, you should state which ones you use.
  \item Suppose $(s_n)$ and $(t_n)$ are two sequences that converge to $s$ and
    $t$ respectively. State the definition of convergence, and use it to prove
    that $3s_n+t_n \to 3s+t$. Do not use the limit theorems for
    sequences.
  \item The Fibonacci numbers are defined by $F_0=0$ and $F_1=1$,
    and
    \[
    F_{n+1} = F_n+F_{n-1}
    \]
    for $n\in \N$. Let the golden ratio be defined as
    $\varphi=\frac{1+\sqrt{5}}{2}$.
    \begin{enumerate}
      \item Show that $\varphi^2=1+\varphi$.
      \item Let
	\[
	f(n)=\frac{\varphi^n-(1-\varphi)^n}{\sqrt{5}}.
	\]
	For $n\in \N$, define $H_n$ to be the hypothesis that ``both
	$F_n=f(n)$ and $F_{n-1}=f(n-1)$''. Apply mathematical induction to
	prove that $H_n$ is true for all $n \in \N$, and deduce that
	$F_n=f(n)$ for all $n \in \N \cup \{0\}$.
	{\it [Hint: it is simpler to carry out the algebra in terms of
	$\varphi$ and use the identity in (a), as opposed to calculating
	explicitly in terms of $(1+\sqrt{5})/2$.]}
      \item Show that $\frac{F_{n+1}}{F_n} \to \varphi$
	as $n \to \infty$.
    \end{enumerate}
  \item Consider two dimensional space $\R^2$, where an element $\vec{x} \in
    \R^2$ is written as $\vec{x}=(x_1,x_2)$. Let $d_E(\vec{x},\vec{y})=(
    (x_1-y_1)^2+(x_2-y_2)^2 )^{1/2}$ be the usual Euclidean metric on $\R^2$.
    \begin{enumerate}
      \item Prove that $d_A(\vec{x},\vec{y}) = \min\{|x_1-y_1|,2|x_2-y_2|\}$ is
	not a metric on $\R^2$.
      \item Prove that $d_B(\vec{x},\vec{y}) = \max\{|x_1-y_1|,2|x_2-y_2|\}$ is
	a metric on $\R^2$. Draw the neighborhood of radius $1$ at $(0,0)$.
      \item Consider an arbitrary metric space $(X,d)$, and a mapping $f: X \to
	\R^2$. Suppose that $f$ is continuous with respect to $(X,d)$ and
	$(\R^2,d_E)$. Prove that it is also continuous with respect to $(X,d)$
	and $(\R^2,d_B)$.
    \end{enumerate}
  \item
    Consider the function defined on $[0,\infty)$ as
    \[
    f(x)= \left\{
    \begin{array}{ll}
      0 & \qquad \textrm{if $0\le x\le1$,} \\
      x & \qquad \textrm{if $x> 1$.}
    \end{array}
    \right.
    \]
    \begin{enumerate}
      \item Compute $F(x)=\int_0^x f(t) dt$ on $[0,\infty)$.
      \item Calculate $F'(x)$, stating the precise range over which it exists.
	You may make use of the second fundamental theorem of calculus.
      \item Prove that neither $f$ nor $F$ is uniformly continuous on $[0,\infty)$.
    \end{enumerate}
  \item Consider the function defined on the domain $[0,\infty)$ as
    \[
    g(x) = \left\{
    \begin{array}{ll}
      x & \qquad \textrm {if $0\le x \le 1$,} \\
      0 & \qquad \textrm{if $x>1$.}
    \end{array}
    \right.
    \]
    Define a sequence of functions on the interval $[0,1]$ according to
    $f_n(x)=n g(nx)$ for $n\in \N$.
    \begin{enumerate}
      \item Sketch $f_1$, $f_2$, and $f_3$ on the interval $[0,1]$.
      \item Prove that $f_n$ converges pointwise to a limit $f$, and determine
	$f$.
      \item Prove that $f_n$ does not converge uniformly to $f$ on $[0,1]$.
      \item Show that $\int_0^1 f_n = \nicefrac{1}{2}$ for all $n$. Does $\int_0^1 f_n$
	converge to $\int_0^1 f$?
    \end{enumerate}

  \item
    \begin{enumerate}
      \item Use L'H\^opital's rule to evaluate
	\[
	\lim_{x\to 0} \frac{x}{e^{x}-e^{-x}}, \qquad
	\lim_{x\to 0} \frac{\sin^2 x}{x^2}.
	\]
      \item Let $f$ be a real-valued function defined on an interval $(a,b)$.
	Let $x\in(a,b)$, and suppose $f$ is twice differentiable at $x$.
	Show that the limit
	\[
	\lim_{h\to 0} \frac{f(x+h)+f(x-h) - 2f(x)}{h^2}
	\]
	exists and equals $f''(x)$.
      \item Construct an example of a real-valued function $f$ defined on an
	interval $(a,b)$, where for some $x\in (a,b)$ the above limit exists
	and is finite, but $f$ is not twice differentiable at $x$.
    \end{enumerate}
  \item Let $(x_n)$ be a sequence. Suppose that the subsequences $(x_{2n})$,
    $(x_{2n+1})$, and $(x_{7n})$ converge. Prove that $(x_n)$ converges.
  \item Let $(a_n)$ be a sequence for $n\in \N$ where
    \[
      a_n= (-1)^n (c+n^{-1})
    \]
    and $c$ is a real constant. Calculate $\lim \sup a_n$ and $\lim \inf
    a_n$ as a function of $c$. For what value of $c$ does $a_n$ converge?
  \item
    \begin{enumerate}
      \item If $f(x)=\log(\log(x))$, calculate $f'$.
      \item Use the integral test to show that
        \[
          \sum_{n=2}^\infty \frac{1}{n \log n}
        \]
        diverges.
      \item Let $s(n)$ be the sum of digits in $n$. For example,
        $s(123)=1+2+3=6$, and $s(99)=9+9=18$. Determine whether
        the series
        \[
          \sum_{n=1}^\infty \frac{1}{n\,s(n)}
        \]
        converges or diverges.
    \end{enumerate}
  \item
    \begin{enumerate}
      \item Suppose $f$ is a continuous function on $[a,b]$, and $f(x)\ge 0$
        for all $x\in[a,b]$. Prove that if $\int_a^b f = 0$, then $f(x)=0$ for
        all $x\in[a,b]$.
      \item Let $X$ be the space of continuous functions on $[a,b]$. Prove
        that for $g,h\in X$, the function
        \[
          d(g,h) = \int_a^b |g(x)-h(x)| dx
        \]
        defines a metric on $X$.
      \item Let $Y$ be the space of integrable functions on $[a,b]$. Show that
        the function $d$ given in part (b) is not a metric on $Y$.
    \end{enumerate}
  \item Define $S=[0,1]$ and let $f_n: S \to \R$ be a sequence of continuous
    functions such that
    \[
      0\le f_{n+1}(x) \le f_n(x) \le \ldots \le f_1(x)
    \]
    for all $n\in \N$ and $x\in S$. Suppose that $(f_n)$ converges pointwise to
    0.
    \begin{enumerate}
      \item For $\epsilon>0$, define the sets $E_n=\{ x\in S \sep f_n(x)
        <\epsilon\}$. Show that for all $n\in \N$, the sets $E_n$ are open in
        $S$ and $E_n \subseteq E_{n+1}$. In addition, show that the
        $\{E_n\}_{n\in \N}$ form an open cover of $S$.
      \item Prove that $(f_n)$ converges uniformly to 0 on $S$.
      \item Suppose now that $S$ is changed to $(0,1)$ with all other
        conditions remaining the same. Prove that $(f_n)$ may not converge
        uniformly to 0 in this case.
    \end{enumerate}
  \item
    \begin{enumerate}
      \item Determine whether the two series
        \[
        \sum_{n=0}^\infty \frac{(-1)^n}{3n+1}, \qquad \sum_{n=0}^\infty \frac{1}{3n+1}
        \]
        converge or diverge.
      \item Determine the radius of convergence of the power series
        \[
          \sum_{n=0}^\infty \frac{(-2)^nx^n}{3n+1}.
        \]
        What is the exact interval of convergence?
    \end{enumerate}
  \item 
    \begin{enumerate}
      \item Prove that the sequence $(s_n)$ is bounded if and only if $\lim \sup |s_n|<\infty$.
      \item Construct an example where $\lim \sup s_n \in \R$ but $(s_n)$ is unbounded.
    \end{enumerate}
  \item
    Let $(f_n)$ be a sequence of increasing differentiable functions on $[0,1]$
    where $f_n(0)=0$ for all $n\in \N$, and $\lim_{n\to \infty} f_n(1) = 0$.
    \begin{enumerate}
      \item Prove that $(f_n)$ converges uniformly to the zero function on $[0,1]$.
      \item Define $g_n = \sup \{ |f'_n(x)| \sep x\in [0,1]\}$. Is $(g_n)$
        always bounded? Prove your assertion.
    \end{enumerate}
  \item Let $f(x) = x^2(1-x)$ and $g(x) = |f(x)|$ for $x\in \R$.
    \begin{enumerate}
      \item Plot $f$ and $g$ over the range $[-1,2]$.
      \item By using the definition of differentiability, prove that $g(x)$ is
	differentiable at $x=0$, but not at $x=1$.
      \item Compute the derivatives $g^{(n)}(2)$ for $n\in \N$, and use them
	to write down the Taylor series of $g$ at $x=2$. Prove that this series
	is equal to $-f(x)$ for all $x\in \R$.
    \end{enumerate}
  \item Let $f$ be a continuous strictly increasing function defined on $\R$.
    Consider a subset $S\subseteq \R$ and let $T=\{f(x) \sep x\in S\}$.
    \begin{enumerate}
      \item If $\sup S$ is finite, prove that $\sup T = f(\sup S)$.
      \item Suppose $(a_n)$ is a sequence such that $\lim \sup a_n$ is finite. Define
	the sequence $(b_n)$ so that $b_n=f(a_n)$ for all $n$. Prove that $\lim
	\sup b_n = f(\lim \sup a_n)$.
      \item Suppose the condition that $f$ is continuous is removed. Construct
	an example of a strictly increasing function $f$ defined on $\R$, and
	subset $S\subseteq \R$, where $\sup S$ is finite, but $\sup T \ne
	f(\sup S)$.
    \end{enumerate}
  \item Define the sequence of functions $(h_n)$ on $\R$
    according to
    \[
    h_n(x) = \left\{
    \begin{array}{ll}
      n & \qquad\text{if $|x|\le\nicefrac{1}{2n}$,} \\
      0 & \qquad\text{if $|x|>\nicefrac{1}{2n}$.}
    \end{array}
    \right.
    \]
    \begin{enumerate}
      \item Sketch $h_1$, $h_2$, and $h_3$.
      \item Prove that $(h_n)$ converges pointwise to $0$ on $\R \setminus \{0\}$.
      \item Let $f$ be a continuous real-valued function on $\R$.
	Prove that
	\[
	\lim_{n\to\infty} \int_{-\infty}^\infty h_n f = f(0).
	\]
      \item Construct an example of an integrable function $g$ on $\R$
	where
	\[
	\lim_{n\to\infty} \int_{-\infty}^\infty h_n g
	\]
	exists and is a real number, but does not equal $g(0)$.
    \end{enumerate}
  \item 
    \begin{enumerate}
      \item Define $B=[1,2]\cup(3,4)$ in $\R$ with the standard metric. Calculate
        the closure $\bar{B}$, and interior $B^\circ$. In addition, calculate
        the boundary defined as $\p B = \bar{B} \setminus B^\circ$.
      \item Let $\{A_n\}_{n=1}^\infty$ be a sequence of open sets in a general
        metric space $(S,d)$ such that $A_{n+1} \supseteq A_n$ for all $n$.
        Let $(x_n)$ be a sequence where $x_n \in A_{n+1} \setminus A_n$, and
        $\lim_{n\to\infty} x_n = x$. Define $A=\bigcup_{n=1}^\infty A_n$. Prove
        that $x \in \p A$.
    \end{enumerate}    
\end{enumerate}
\end{document}
