
\documentclass[12pt]{article}

% Essential math packages
\usepackage{amsmath}
\usepackage{amssymb}
\usepackage{amsthm}
\usepackage{mathtools}

% Other useful packages
\usepackage{geometry}
\usepackage{graphicx}
\usepackage{enumitem}

% Page margins
\geometry{margin=1in}


% Title information
\title{The Brachistrochrone Problem: a Finite Element Approach}
\author{Harry Luo}


\begin{document}

\maketitle
\begin{abstract}
    The Brachistochrone problem seeks the path \(y(x)\) between two points that allows a particle sliding under gravity to travel in the minimum possible time. We present a numerical solution using the Finite Element Analysis (FEA) method. The continuous problem, formulated as minimizing the time functional \[ T[y(x)] = \int_0^{x_f} \frac{\sqrt{1 + (y'(x))^2}}{\sqrt{2gy(x)}} dx, \] is discretized using P2 quadratic elements. This transforms the variational problem into a finite-dimensional nonlinear optimization problem, which is then solved using the L-BFGS-B algorithm.
\end{abstract}
    
\section{Finite Element Discretization}
    
    We employ quadratic (P2) finite elements to discretize the time functional and approximate the unknown path \(y(x)\).
    
    First, we divide the domain \( \left[0, x_f\right] \) into \(N \) uniform finite elements, each of length \( h = x_f / N \). We define a total of \(2N+1\) global nodes along the domain. The coordinate of the \(m\)-th global node (where \(m = 0, 1, \ldots, 2N\)) is given by
    \[ x_m = m \frac{h}{2}. \]
    Note that nodes with even indices \(m=2n\) ($n=0..N$) lie at the boundaries between elements (or domain ends), while nodes with odd indices \(m=2n+1\) ($n=0..N-1$) lie at the midpoints of the elements.
    
    An element \(e\) (where \(e = 1, 2, \ldots, N\)) is defined by three consecutive global nodes: a start node \(i = 2(e-1)\), a midpoint node \(j = 2e-1\), and an end node \(k = 2e\). The element thus spans the physical interval \( [x_i, x_k] \).
    
    We seek to determine the approximate vertical position \( y \) at each global node \(m\). Let \( y_m \approx y(x_m) \) denote the nodal value at node \(m\). These \( y_m \) values are the fundamental variables in our discretized problem. The boundary conditions fix the values at the first and last nodes:
    \[ y_{0} = y(x_{0}) = y(0) = 0 \quad , \quad y_{2N} = y(x_{2N}) = y(x_f) = y_f. \]
    The actual unknowns to be solved for are the values at the interior nodes ($m = 1, 2, \ldots, 2N-1$). We collect these unknown nodal values into the vector of degrees of freedom:
    \[ \mathbf{y_\text{int}} = \left[ y_1, y_2, \ldots, y_{2N-1} \right]^T \in \mathbb{R}^{2N-1}. \]
    % Note: In this context, \mathbf{y} refers to the vector of unknowns, not the full set of nodal values.
    
    %%%%%%%%%%%%%%%%%%%%%%%%%%%%%%%
    
\subsection{Local Coordinate System}
    
    To define approximations consistently within each element, it is convenient to map the physical coordinates \( x \) belonging to an element \(e\) (i.e., \(x \in [x_i, x_k]\)) to a dimensionless local coordinate \( \xi \in \left[-1,1\right] \). The mapping places the local origin \( \xi = 0 \) at the element's midpoint node \( x_j \):
    \[
        x(\xi) := \frac{x_{i} + x_k}{2} + \frac{x_k - x_{i}}{2} \xi =  x_j + \frac{h}{2} \xi.
    \]
    Here, the local coordinate \( \xi = -1 \) corresponds to the element's start node \( x_{i} \), \( \xi = 0 \) corresponds to the midpoint node \( x_j \), and \( \xi = 1 \) corresponds to the end node \( x_k \).
    
    The Jacobian of this transformation relates the physical and local differentials:
    \[ J = \frac{dx}{d\xi} = \frac{x_k - x_i}{2} = \frac{h}{2}. \]
    Hence, the differential transformation is
    \[
        dx = J d\xi = \frac{h}{2} d\xi.
    \]
    
    %%%%%%%%%%%%%%%%%%%%%%%%%%%%%%%%%%%%%%%%%%%%%%
    
    \subsection{Quadratic Shape Functions for Approximating \( y(x) \) and \( y'(x) \) }
    
    Within element \(e\), let the nodal values corresponding to its start, mid, and end nodes be collected in the element nodal vector \( \mathbf{y}_{e} = \left[y_{i}, y_{j}, y_k\right]^{T}\), where \(i, j, k\) are the global indices for element \(e\). We approximate the function \( y(x) \) within this element using an interpolation \( y_h (x) \) based on these nodal values and quadratic shape functions of the local coordinate \( \xi \):
    \[
        y_h \left(\,x(\xi) \right) =  N_{1}(\xi) y_{i} + N_{2}(\xi) y_{j} + N_{3}(\xi) y_k.
    \]
    The quadratic shape functions \( N_1(\xi), N_2(\xi), N_3(\xi) \) must satisfy the interpolation property
    \[
        N_m(\xi_n) = \delta_{mn} \quad \text{for } m, n = 1, 2, 3,
    \]
    where we associate local node numbers $1, 2, 3$ with local coordinates
    \( \xi_1 = -1, \quad \xi_2 = 0, \quad \xi_3 = 1, \)
    and where \(\delta_{mn}\) denotes the Kronecker delta. This property ensures that the approximation \(y_h\) exactly matches the nodal values at the element's nodes: \( y_h(x_i) = y_i\), \( y_h(x_j) = y_j\), \( y_h(x_k) = y_k \).
    
    The unique quadratic polynomials satisfying these conditions are the Lagrange polynomials on \([-1, 1]\) for nodes at \(-1, 0, 1\):
    \[
    \begin{aligned}
        N_1(\xi) &= \frac{(\xi - 0)(\xi - 1)}{(-1 - 0)(-1 - 1)} = \frac{1}{2}\xi(\xi-1), \\
        N_2(\xi) &= \frac{(\xi - (-1))(\xi - 1)}{(0 - (-1))(0 - 1)} = \frac{(\xi+1)(\xi-1)}{-1} = 1 - \xi^2 ,\\
        N_3(\xi) &= \frac{(\xi - (-1))(\xi - 0)}{(1 - (-1))(1 - 0)} = \frac{1}{2}\xi(\xi+1).
    \end{aligned}
    \]
    The element approximation can be written compactly using vector notation:
    \[
        y_h\left(x(\xi)\right) = \mathbf{N}(\xi) \cdot \mathbf{y}_e,
    \]
    where \( \mathbf{N}(\xi) = [N_1(\xi), \; N_2(\xi), \; N_3(\xi)] \) is the vector of shape functions.
    
    Since the time functional \( T[y(x)] \) also depends on the derivative \( y'(x) \), we approximate this with the derivative of the interpolation, \( y'_h (x) \). Using the chain rule:
    \[
        y'_h (x) = \frac{dy_h}{dx}  = \frac{dy_h}{d\xi} \frac{d\xi}{dx}.
    \]
    The derivative with respect to the local coordinate is
    \[
        \frac{dy_h}{d\xi} = \frac{d}{d\xi} (\mathbf{N}(\xi) \cdot \mathbf{y}_e) = \left( \frac{d\mathbf{N}}{d\xi} \right) \cdot \mathbf{y}_e,
    \]
    where \( \frac{d\mathbf{N}}{d\xi} = \left[ \xi - \frac{1}{2},\; -2\xi,\; \xi + \frac{1}{2} \right] \).
    Using the inverse Jacobian \( \frac{d\xi}{dx} = 1/J = 2/h \), the approximation for the physical derivative becomes:
    \[
        y'_h(x(\xi)) = \left( \frac{d\mathbf{N}}{d\xi} \cdot \mathbf{y}_e \right) \frac{2}{h} = \frac{2}{h} \left( \left[ \xi - \frac{1}{2},\; -2\xi,\; \xi + \frac{1}{2} \right] \mathbf{y}_e \right).
    \]
    
    \subsection{Discretized Functional}
    
    For numerical optimization, we minimize the functional \( T^* \) which omits the constant factor \( 1/\sqrt{2g} \) from the original total time functinoal:
    \[
        T^*[y(x)] = \int_0^{x_f} \sqrt{\frac{1 + (y'(x))^2}{y(x)}} dx .
    \]
    We approximate this functional by replacing the exact function \( y(x) \) and its derivative \( y'(x) \) with their finite element approximations \( y_h (x) \) and \( y'_h (x) \). The integral over the full domain \( [0, x_f] \) is computed as the sum of integrals over each element \(e\):
    \[
        T^*[y(x)] \approx T_h(\mathbf{y}) = \sum_{e=1}^{N} T_e^*(\mathbf{y}_e)
    \]
    where \( T_e^* \) is the contribution from element \(e\), whose associated global nodes are \(i, j, k\):
    \[
        T_e^*(\mathbf{y}_e) = \int_{x_{i}}^{x_{k}} \sqrt{\frac{1 + (y'_h(x))^2}{y_h(x)}} dx.
    \]
    The total approximate functional \( T_h \) is now expressed solely in terms of the nodal values \(\mathbf{y}_m\), specifically the unknown ones contained in the vector \(\mathbf{y}\). The next step involves evaluating these element integrals \(T_e^*\) using numerical quadrature.




    \end{document}