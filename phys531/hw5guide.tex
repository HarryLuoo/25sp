\documentclass[10pt]{article}
\usepackage{amsmath, amssymb, amsthm}
\usepackage{geometry}
\usepackage{graphicx}
\geometry{letterpaper, margin=0.7in}

\title{Homework 5 Guide - Advanced Quantum Mechanics}
\date{Due: 04/13/2025}
\author{Solution Strategy Guide}

\begin{document}
\maketitle

% %%%%%%%%%%%%%%%%%%%%%%%%%%%%%%%%%%%%%%%%%%%%%%%%%%%%%%% %
%                   Problem 1                             %
% %%%%%%%%%%%%%%%%%%%%%%%%%%%%%%%%%%%%%%%%%%%%%%%%%%%%%%% %
\section*{Problem 1: Particle in a box}

\subsection*{Note}

\subsubsection*{Background Knowledge}
This problem contrasts the behavior of a free particle with that of a particle confined to an infinite potential well (a "box").
\begin{itemize}
    \item The dynamics are governed by the Time-Independent Schrödinger Equation (TISE): \(\hat{H}\psi(x) = E\psi(x)\).
    \item For a 1D particle, the Hamiltonian is \(\hat{H} = \frac{\hat{p}^2}{2m} + V(\hat{x})\), with \(\hat{p} = -i\hbar \frac{d}{dx}\).
    \item A free particle has \(V(x) = 0\), leading to continuous energy eigenvalues \(E > 0\).
    \item The infinite potential well is defined by \(V(x) = 0\) for \(|x| < L/2\) and \(V(x) = \infty\) for \(|x| \ge L/2\).
    \item The requirement that the wave function \(\psi(x)\) be continuous, combined with the infinite potential, imposes boundary conditions: \(\psi(x) = 0\) at the edges of the well (\(x = \pm L/2\)).
    \item These boundary conditions lead to energy quantization (discrete energy levels) and specific forms for the eigenstates (standing waves).
    \item We compare the results (energy spectrum, wave function nodes) to the quantum harmonic oscillator (QHO), characterized by \(V(x) = \frac{1}{2}m\omega^2 x^2\) and energy levels \(E_n^{HO} = \hbar\omega(n+1/2)\).
\end{itemize}

\subsubsection*{Key Equations}
\begin{itemize}
    \item Free particle TISE (\(E>0\)): \(-\frac{\hbar^2}{2m}\frac{d^2\psi}{dx^2} = E\psi\).
    \item General solution for free particle (\(E>0\)): \(\psi(x) = A\sin(k_E x) + B\cos(k_E x)\), where \(k_E = \sqrt{2mE}/\hbar\).
    \item Infinite potential well definition: \(V(x) = \begin{cases} \infty, & |x| \geq L/2 \\ 0, & |x|<L/2 \end{cases}\).
    \item Boundary Conditions (BCs) for the well: \(\psi(-L/2) = \psi(L/2) = 0\).
    \item Energy levels in the well: \(E_n = \frac{\hbar^2 \pi^2 n^2}{2mL^2}\), for \(n=1, 2, 3, \dots\).
    \item Normalized eigenstates in the well:
    \[ \psi_n(x) = \sqrt{\frac{2}{L}} \times \begin{cases} \cos(\frac{n\pi x}{L}) & \text{if } n \text{ is odd} \\ \sin(\frac{n\pi x}{L}) & \text{if } n \text{ is even} \end{cases} \quad \text{for } |x| < L/2, \text{ and } 0 \text{ otherwise.} \]
    \item QHO Energy levels: \(E_n^{HO} = \hbar\omega(n+1/2)\), for \(n=0, 1, 2, \dots\).
    \item Nodes: Points where \(\psi(x)=0\) (excluding boundaries for the box).
\end{itemize}

\subsubsection*{Solution Strategy}

\paragraph{Part a): Free Particle}
\begin{enumerate}
    \item Write the TISE for \(V(x)=0\).
    \item Substitute the proposed wave function \(\psi(x) = A \sin(k_E x) + B \cos(k_E x)\) into the TISE.
    \item Calculate the second derivative \(\frac{d^2\psi}{dx^2}\).
    \item Show that the TISE is satisfied if and only if \(k_E^2 = 2mE/\hbar^2\), confirming the form of \(k_E\).
\end{enumerate}

\paragraph{Part b): Particle in a Box}
\begin{enumerate}
    \item[(i)] \textbf{Energy Spectrum and Eigenstates:}
        \begin{itemize}
            \item Inside the well (\(|x| < L/2\)), the potential is zero, so the general solution form \(A\sin(kx) + B\cos(kx)\) still applies, with \(k=\sqrt{2mE}/\hbar\).
            \item Apply the BC \(\psi(L/2)=0\): \(A\sin(kL/2) + B\cos(kL/2) = 0\).
            \item Apply the BC \(\psi(-L/2)=0\): \(-A\sin(kL/2) + B\cos(kL/2) = 0\).
            \item Solve this system of linear equations for \(A\) and \(B\). Show that non-trivial solutions exist only if \(A=0\) and \(\cos(kL/2)=0\) (leading to even states) or \(B=0\) and \(\sin(kL/2)=0\) (leading to odd states).
            \item Combine these conditions to find the allowed values of \(k\), denoted \(k_n = n\pi/L\) for \(n=1, 2, 3, \dots\).
            \item Derive the quantized energy levels \(E_n = \frac{\hbar^2 k_n^2}{2m}\).
            \item Write down the corresponding unnormalized wave functions, distinguishing between odd \(n\) (cosine solutions, even parity) and even \(n\) (sine solutions, odd parity).
            \item Normalize the wave functions over the interval \([-L/2, L/2]\) by calculating \(\int_{-L/2}^{L/2} |\psi_n(x)|^2 dx = 1\).
        \end{itemize}
    \item[(ii)] \textbf{Plot and Compare Spectrum:}
        \begin{itemize}
            \item Sketch the energy levels \(E_n \propto n^2\) for the infinite well.
            \item Sketch the energy levels \(E_n^{HO} \propto (n+1/2)\) for the QHO.
            \item Compare the dependence on the quantum number \(n\) and the spacing between adjacent energy levels. Note that the ground state index is \(n=1\) for the box and \(n=0\) for the HO.
        \end{itemize}
    \item[(iii)] \textbf{Compare Sign Changes (Nodes):}
        \begin{itemize}
            \item For the box, determine the number of points within the open interval \((-L/2, L/2)\) where \(\psi_n(x) = 0\) for the ground state (\(n=1\)) and the first few excited states (\(n=2, 3, \dots\)). Identify the general pattern relating the number of nodes to \(n\).
            \item Recall (or derive using results from Problem 4) the number of nodes for the QHO ground state (\(n=0\)) and the first few excited states (\(n=1, 2, \dots\)).
            \item Compare the number of nodes for the \(k\)-th excited state in both systems (remembering that the \(k\)-th excited state corresponds to \(n=k+1\) for the box and \(n=k\) for the QHO).
        \end{itemize}
\end{enumerate}

% %%%%%%%%%%%%%%%%%%%%%%%%%%%%%%%%%%%%%%%%%%%%%%%%%%%%%%% %
%                   Problem 2                             %
% %%%%%%%%%%%%%%%%%%%%%%%%%%%%%%%%%%%%%%%%%%%%%%%%%%%%%%% %
\section*{Problem 2: Spherical Harmonics}

\subsection*{Note}

\subsubsection*{Background Knowledge}
Spherical harmonics \(Y_l^m(\theta, \phi)\) are fundamental functions in quantum mechanics for systems with spherical symmetry.
\begin{itemize}
    \item They are eigenfunctions of the angular momentum operators \(\hat{L}^2\) and \(\hat{L}_z\).
    \item They form a complete orthonormal basis on the surface of a sphere.
    \item The indices \(l\) (orbital) and \(m\) (magnetic) are integers with \(l \ge 0\) and \(-l \le m \le l\).
    \item They are composed of an azimuthal part \(e^{im\phi}\) and a polar part involving associated Legendre polynomials \(P_l^m(\cos\theta)\).
    \item Legendre polynomials \(P_l(x)\) (\(x = \cos\theta\)) correspond to the case \(m=0\). They are orthogonal on the interval \([-1, 1]\).
\end{itemize}

\subsubsection*{Key Equations}
\begin{itemize}
    \item Spherical Harmonic definition: \(Y_{l}^{m}(\theta, \phi)=\mathcal{N}_{lm} e^{i m \phi} P_{l}^{m}(\cos (\theta))\).
    \item Normalization constant: \(\mathcal{N}_{lm} = \sqrt{\frac{(2l+1)}{4\pi} \frac{(l-m)!}{(l+m)!}}\).
    \item Associated Legendre Polynomial (definition from problem): \(P_{l}^{m}(x)=(-1)^{m}\left(1-x^{2}\right)^{m / 2} \frac{d^{m}}{d x^{m}} P_{l}(x)\) (for \(m\ge 0\)).
    \item Legendre Polynomial Recursion: \((l+1) P_{l+1}(x)=(2 l+1) x P_{l}(x)-l P_{l-1}(x)\), \(P_0(x)=1, P_1(x)=x\).
    \item Spherical Harmonic Orthogonality:
    \[ \int_{0}^{\pi} \sin\theta d\theta \int_{0}^{2\pi} d\phi [Y_{l}^{m}(\theta, \phi)]^{*} Y_{l^{\prime}}^{m^{\prime}}(\theta, \phi) = \int_{-1}^{1} dx \int_{0}^{2\pi} d\phi [Y_{l}^{m}]^{*} Y_{l^{\prime}}^{m^{\prime}} = \delta_{ll'} \delta_{mm'} \]
    (with \(x = \cos\theta\)).
    \item Legendre Polynomial Orthogonality: \(\int_{-1}^{1} dx P_l(x) P_{l'}(x) = \frac{2}{2l+1} \delta_{ll'}\).
\end{itemize}

\subsubsection*{Solution Strategy}

\paragraph{Part a): Orthogonality in \(m\)}
\begin{enumerate}
    \item Write the orthogonality integral in terms of \(\theta\) and \(\phi\), then change variables to \(x = \cos\theta\).
    \item Substitute the definition \(Y_{l}^{m}(\theta, \phi)=\mathcal{N}_{lm} e^{i m \phi} P_{l}^{m}(x)\) into the integral.
    \item Separate the integral into a part over \(x\) (involving \(P_l^m\)) and a part over \(\phi\) (involving \(e^{im\phi}\)).
    \item Evaluate the \(\phi\) integral: \( \int_{0}^{2\pi} d\phi (e^{im\phi})^* (e^{im'\phi}) = \int_{0}^{2\pi} d\phi e^{i(m'-m)\phi} \).
    \item Show that this integral equals \(2\pi\) if \(m=m'\) and \(0\) if \(m \neq m'\).
    \item Conclude that the entire integral is zero when \(m \neq m'\).
\end{enumerate}

\paragraph{Part b): Legendre Polynomials Orthogonality}
\begin{enumerate}
    \item Use the Legendre recursion relation, starting with \(P_0(x)=1\) and \(P_1(x)=x\), to explicitly calculate \(P_2(x)\) and \(P_3(x)\).
    \item Select pairs \((l, l')\) from \(\{0, 1, 2, 3\}\) with \(l \neq l'\) and explicitly calculate the integral \(\int_{-1}^{1} dx P_l(x) P_{l'}(x)\). Show it evaluates to zero (e.g., calculate \(\int P_0 P_1 dx\), \(\int P_1 P_2 dx\), \(\int P_0 P_3 dx\), etc.).
    \item Select pairs with \(l = l'\) and calculate \(\int_{-1}^{1} dx [P_l(x)]^2\). Show it is non-zero.
    \item Relate these results to the general orthogonality property \(\int_{-1}^{1} dx P_l(x) P_{l'}(x) \propto \delta_{ll'}\). You can explicitly check the proportionality constant \(2/(2l+1)\) for the calculated cases.
\end{enumerate}

\paragraph{Part c): Specific \(Y_l^m\) Calculation}
\begin{enumerate}
    \item \textbf{Calculate \(Y_3^0(\theta, \phi)\):}
        \begin{itemize}
            \item Identify \(l=3, m=0\). Use \(P_3^0(x) = P_3(x)\) found in part (b).
            \item Calculate the normalization constant \(\mathcal{N}_{30}\).
            \item Combine the parts: \(Y_3^0 = \mathcal{N}_{30} e^{i0\phi} P_3(\cos\theta)\). Substitute \(x = \cos\theta\).
        \end{itemize}
    \item \textbf{Calculate \(Y_3^3(\theta, \phi)\):}
        \begin{itemize}
            \item Identify \(l=3, m=3\). Use the formula \(P_{l}^{m}(x)=(-1)^{m}\left(1-x^{2}\right)^{m / 2} \frac{d^{m}}{d x^{m}} P_{l}(x)\) to find \(P_3^3(x)\).
            \item This requires calculating the third derivative of \(P_3(x)\).
            \item Calculate the normalization constant \(\mathcal{N}_{33}\). Remember \(0!=1\).
            \item Combine the parts: \(Y_3^3 = \mathcal{N}_{33} e^{i3\phi} P_3^3(\cos\theta)\). Substitute \(x=\cos\theta\) and \(1-x^2 = \sin^2\theta\). Simplify the resulting expression. Be careful with signs coming from \( (-1)^m \).
        \end{itemize}
\end{enumerate}

% %%%%%%%%%%%%%%%%%%%%%%%%%%%%%%%%%%%%%%%%%%%%%%%%%%%%%%% %
%                   Problem 3                             %
% %%%%%%%%%%%%%%%%%%%%%%%%%%%%%%%%%%%%%%%%%%%%%%%%%%%%%%% %
\section*{Problem 3: Algebra of harmonic oscillator}

\subsection*{Note}

\subsubsection*{Background Knowledge}
The algebraic method provides an elegant way to solve the QHO using ladder operators (\(\hat{a}\) and \(\hat{a}^\dagger\)).
\begin{itemize}
    \item These operators are constructed from \(\hat{x}\) and \(\hat{p}\).
    \item Their commutation relation \([\hat{a}, \hat{a}^\dagger] = 1\) is fundamental.
    \item The number operator \(\hat{N} = \hat{a}^\dagger \hat{a}\) commutes with the Hamiltonian and has eigenstates \(|n\rangle\) with integer eigenvalues \(n=0, 1, 2, \dots\).
    \item \(\hat{a}\) lowers the eigenvalue \(n\) by 1 (\(\hat{a}|n\rangle \propto |n-1\rangle\)), while \(\hat{a}^\dagger\) raises it by 1 (\(\hat{a}^\dagger|n\rangle \propto |n+1\rangle\)).
    \item The ground state \(|0\rangle\) is defined by \(\hat{a}|0\rangle = 0\).
    \item The uncertainty principle \(\Delta x \Delta p \ge \hbar/2\) provides a lower bound on the product of uncertainties.
\end{itemize}

\subsubsection*{Key Equations}
\begin{itemize}
    \item Operators: \(\hat{a}=\frac{1}{\sqrt{2} \ell}\left(\hat{x}+\frac{i \ell^{2}}{\hbar} \hat{p}\right), \quad \hat{a}^{\dagger}=\frac{1}{\sqrt{2} \ell}\left(\hat{x}-\frac{i \ell^{2}}{\hbar} \hat{p}\right)\), with \(\ell=\sqrt{\frac{\hbar}{m \omega}}\).
    \item Number operator: \(\hat{N}=\hat{a}^{\dagger} \hat{a}\).
    \item CCR: \([\hat{x}, \hat{p}]=i\hbar\).
    \item Inverse relations: \(\hat{x} = \frac{\ell}{\sqrt{2}}(\hat{a}+\hat{a}^\dagger), \quad \hat{p} = \frac{\hbar}{i\ell\sqrt{2}}(\hat{a}-\hat{a}^\dagger)\).
    \item Key commutators: \([\hat{a}, \hat{a}^\dagger]=1\), \([\hat{a}, \hat{N}]=\hat{a}\), \([\hat{a}^\dagger, \hat{N}]=-\hat{a}^\dagger\).
    \item Action on number states: \(\hat{a}|n\rangle=\sqrt{n}|n-1\rangle\), \(\hat{a}^{\dagger}|n\rangle=\sqrt{n+1}|n+1\rangle\).
    \item Orthonormality: \(\langle m | n \rangle = \delta_{mn}\).
    \item Standard deviation: \(\Delta O = \sqrt{\langle \hat{O}^2 \rangle - \langle \hat{O} \rangle^2}\).
\end{itemize}

\subsubsection*{Solution Strategy}

\paragraph{Part a): Basic Algebra}
\begin{enumerate}
    \item[i)] Calculate \([\hat{a}, \hat{a}^\dagger]\ = \hat{a}\hat{a}^\dagger - \hat{a}^\dagger\hat{a}\). Substitute the definitions of \(\hat{a}, \hat{a}^\dagger\) in terms of \(\hat{x}, \hat{p}\). Expand the product and use \([\hat{x}, \hat{p}]=i\hbar\). Simplify the result to 1.
    \item[ii)] Calculate \(\{\hat{a}, \hat{a}^\dagger\} = \hat{a}\hat{a}^\dagger + \hat{a}^\dagger\hat{a}\). Express \(\hat{x}\) and \(\hat{p}\) in terms of \(\hat{a}, \hat{a}^\dagger\). Substitute these into \(\frac{\hat{x}^2}{\ell^2} + \frac{\ell^2 \hat{p}^2}{\hbar^2}\). Expand the terms \(\hat{x}^2\) and \(\hat{p}^2\) using \(\hat{a}, \hat{a}^\dagger\). Show that the sum simplifies to \(\hat{a}\hat{a}^\dagger + \hat{a}^\dagger\hat{a}\). (Alternatively, express \(\{\hat{a}, \hat{a}^\dagger\}\) in terms of \(\hat{x}, \hat{p}\) and simplify).
    \item[iii)] Calculate \([\hat{a}, \hat{N}] = [\hat{a}, \hat{a}^\dagger \hat{a}]\). Use the commutator identity \([A, BC]=[A,B]C + B[A,C]\) and the known values \([\hat{a}, \hat{a}^\dagger]=1\) and \([\hat{a}, \hat{a}]=0\). Simplify the result to \(\hat{a}\).
\end{enumerate}

\paragraph{Part b): Action on Number States}
\begin{enumerate}
    \item[i)] To show \(\hat{a}|n\rangle=\sqrt{n}|n-1\rangle\):
        \begin{itemize}
            \item First, show that \(\hat{a}|n\rangle\) is an eigenstate of \(\hat{N}\) with eigenvalue \(n-1\). Use \([\hat{a}, \hat{N}]=\hat{a}\) to write \(\hat{N}\hat{a} = \hat{a}\hat{N} - \hat{a}\) and apply this to \(|n\rangle\).
            \item Second, calculate the norm squared of \(\hat{a}|n\rangle\): \(||\hat{a}|n\rangle||^2 = \langle n | \hat{a}^\dagger \hat{a} | n \rangle = \langle n | \hat{N} | n \rangle = n\).
            \item Conclude that \(\hat{a}|n\rangle\) must be \(c|n-1\rangle\) where \(|c|^2=n\). Use the standard phase convention \(c=\sqrt{n}\). Address the \(n=0\) case separately: \(\hat{a}|0\rangle = 0\).
        \end{itemize}
    \item[ii)] To show \(\hat{a}^\dagger|n\rangle=\sqrt{n+1}|n+1\rangle\):
        \begin{itemize}
            \item Show that \(\hat{a}^\dagger|n\rangle\) is an eigenstate of \(\hat{N}\) with eigenvalue \(n+1\). Calculate \([\hat{a}^\dagger, \hat{N}] = -\hat{a}^\dagger\) (similar to part a.iii), which gives \(\hat{N}\hat{a}^\dagger = \hat{a}^\dagger\hat{N} + \hat{a}^\dagger\). Apply this to \(|n\rangle\).
            \item Calculate the norm squared: \(||\hat{a}^\dagger|n\rangle||^2 = \langle n | \hat{a} \hat{a}^\dagger | n \rangle\). Use \([\hat{a}, \hat{a}^\dagger]=1 \implies \hat{a}\hat{a}^\dagger = \hat{a}^\dagger\hat{a} + 1 = \hat{N} + 1\). Evaluate \(\langle n|\hat{N}+1|n\rangle = n+1\).
            \item Conclude \(\hat{a}^\dagger|n\rangle = d|n+1\rangle\) with \(|d|^2 = n+1\). Use convention \(d=\sqrt{n+1}\).
        \end{itemize}
    \item[iii)] Deduce orthogonality \(\langle m | n \rangle = 0\) for \(n > m\).
        \begin{itemize}
            \item Use the fact that \(|n\rangle\) and \(|m\rangle\) are eigenstates of the Hermitian operator \(\hat{N}\) with different eigenvalues \(n\) and \(m\). Recall the theorem that eigenstates of a Hermitian operator corresponding to distinct eigenvalues are orthogonal.
            \item Alternatively, use \(|n\rangle = \frac{(\hat{a}^\dagger)^n}{\sqrt{n!}}|0\rangle\) and \(|m\rangle = \frac{(\hat{a}^\dagger)^m}{\sqrt{m!}}|0\rangle\). Write \(\langle m|n\rangle = \frac{1}{\sqrt{n!m!}} \langle 0 | (\hat{a})^m (\hat{a}^\dagger)^n | 0 \rangle\). Use \(\hat{a}|0\rangle=0\) and \([\hat{a}, \hat{a}^\dagger]=1\) to commute the \(m\) operators \(\hat{a}\) to the right until they hit \(|0\rangle\). Since \(n > m\), show that there will always be remaining \(\hat{a}^\dagger\) operators, and the expression evaluates to \(\langle 0 | (\hat{a}^\dagger)^{n-m} | 0 \rangle \times \text{constant}\). However, this needs careful handling of commutations. The eigenvalue argument is simpler. The structure \((\hat{a}^\dagger)^n|0\rangle\) confirms they are eigenstates. Focus on the eigenvalue argument.
        \end{itemize}
\end{enumerate}

\paragraph{Part d): Uncertainty Product}
\begin{enumerate}
    \item Calculate the expectation values \(\langle \hat{x} \rangle_n = \langle n | \hat{x} | n \rangle\) and \(\langle \hat{p} \rangle_n = \langle n | \hat{p} | n \rangle\). Express \(\hat{x}\) and \(\hat{p}\) in terms of \(\hat{a}, \hat{a}^\dagger\). Use orthogonality \(\langle n|n\pm 1\rangle=0\).
    \item Calculate \(\langle \hat{x}^2 \rangle_n = \langle n | \hat{x}^2 | n \rangle\) and \(\langle \hat{p}^2 \rangle_n = \langle n | \hat{p}^2 | n \rangle\). Express \(\hat{x}^2\) and \(\hat{p}^2\) in terms of \(\hat{a}, \hat{a}^\dagger\). Expand the products (e.g., \((\hat{a}+\hat{a}^\dagger)^2 = \hat{a}^2 + \hat{a}\hat{a}^\dagger + \hat{a}^\dagger\hat{a} + (\hat{a}^\dagger)^2\)). Use \(\langle n | \hat{a}^2 | n \rangle = 0\), \(\langle n | (\hat{a}^\dagger)^2 | n \rangle = 0\), \(\langle n | \hat{a}\hat{a}^\dagger | n \rangle = n+1\), \(\langle n | \hat{a}^\dagger\hat{a} | n \rangle = n\).
    \item Calculate the variances \((\Delta x)_n^2 = \langle \hat{x}^2 \rangle_n - \langle \hat{x} \rangle_n^2\) and \((\Delta p)_n^2 = \langle \hat{p}^2 \rangle_n - \langle \hat{p} \rangle_n^2\).
    \item Find the standard deviations \(\Delta x_n\) and \(\Delta p_n\).
    \item Calculate the product \(\Delta x_n \Delta p_n\).
    \item Compare the result with the Heisenberg Uncertainty Principle inequality \(\Delta x \Delta p \ge \hbar/2\). Check if the QHO eigenstates satisfy it, and identify if/when they saturate the bound.
\end{enumerate}


% %%%%%%%%%%%%%%%%%%%%%%%%%%%%%%%%%%%%%%%%%%%%%%%%%%%%%%% %
%                   Problem 4                             %
% %%%%%%%%%%%%%%%%%%%%%%%%%%%%%%%%%%%%%%%%%%%%%%%%%%%%%%% %
\section*{Problem 4: Hermite polynomials and the harmonic oscillator}

\subsection*{Note}

\subsubsection*{Background Knowledge}
This problem connects the abstract algebraic solution (\(|n\rangle\) states) to the concrete wave functions \(\psi_n(x) = \langle x | n \rangle\) by solving the TISE in the position representation.
\begin{itemize}
    \item The TISE for the QHO is a second-order ordinary differential equation.
    \item Changing to a dimensionless coordinate \(y=x/\ell\) simplifies the equation.
    \item The asymptotic behavior of the solutions (\(x\to\pm\infty\)) suggests factoring out a Gaussian term \(e^{-y^2/2}\).
    \item The remaining part satisfies Hermite's differential equation.
    \item The polynomial solutions to Hermite's equation are the Hermite polynomials \(H_n(y)\).
    \item Hermite polynomials can be defined via Rodrigues' formula and are orthogonal with respect to the weight function \(e^{-y^2}\) over \((-\infty, \infty)\).
\end{itemize}

\subsubsection*{Key Equations}
\begin{itemize}
    \item QHO Hamiltonian: \(\hat{H}=\frac{\hat{p}^{2}}{2 m}+\frac{m \omega^{2}}{2} \hat{x}^{2}\).
    \item TISE: \(\hat{H} \psi_{n}(x)=E_{n} \psi_{n}(x)\) with \(E_n = \hbar\omega(n+1/2)\).
    \item Position representation: \(\hat{p} = -i\hbar\frac{d}{dx}\).
    \item Dimensionless variable: \(y=x/\ell\), \(\ell = \sqrt{\hbar/m\omega}\).
    \item Transformed TISE: \(\left[y^{2}-\frac{d^{2}}{d y^{2}}\right] \phi_{n}(y)=(2 n+1) \phi_{n}(y)\), where \(\phi_n(y) = \psi_n(x)\).
    \item Ansatz: \(\phi_{n}(y)=H_{n}(y) e^{-y^{2} / 2}\).
    \item Hermite Equation: \(H_{n}^{\prime \prime}(y)-2 y H_{n}^{\prime}(y)+2 n H_{n}(y)=0\).
    \item Rodrigues' formula: \(H_{n}(y)=(-1)^{n} e^{y^{2}} \frac{d^{n}}{d y^{n}} e^{-y^{2}}\).
    \item Hermite Polynomial Orthogonality: \(\int_{-\infty}^{\infty} dy e^{-y^{2}} H_{n}(y) H_{m}(y) \propto \delta_{nm}\).
\end{itemize}

\subsubsection*{Solution Strategy}

\paragraph{Part a): Derivation of Hermite's Equation}
\begin{enumerate}
    \item[i)] Transform the TISE to dimensionless form:
        \begin{itemize}
            \item Start with the TISE in \(x\) coordinates: \([-\frac{\hbar^2}{2m}\frac{d^2}{dx^2} + \frac{1}{2}m\omega^2 x^2] \psi_n(x) = E_n \psi_n(x)\).
            \item Substitute \(x = \ell y\), \(\frac{d}{dx} = \frac{1}{\ell}\frac{d}{dy}\), \(\frac{d^2}{dx^2} = \frac{1}{\ell^2}\frac{d^2}{dy^2}\), \(\psi_n(x) = \phi_n(y)\), and \(E_n = \hbar\omega(n+1/2)\).
            \item Use \(\ell^2 = \hbar/m\omega\) to simplify the coefficients of \(\frac{d^2}{dy^2}\) and \(y^2\).
            \item Divide the entire equation by \(\hbar\omega/2\) and rearrange to obtain \(\left[y^{2}-\frac{d^{2}}{d y^{2}}\right] \phi_{n}(y)=(2 n+1) \phi_{n}(y)\).
        \end{itemize}
    \item[ii)] Derive Hermite's Equation from the Ansatz:
        \begin{itemize}
            \item Substitute the Ansatz \(\phi_n(y) = H_n(y) e^{-y^2/2}\) into the equation found in (i), specifically into the form \(\frac{d^2\phi_n}{dy^2} = [y^2 - (2n+1)] \phi_n(y)\).
            \item Calculate the first derivative \(\phi_n'(y)\) using the product rule.
            \item Calculate the second derivative \(\phi_n''(y)\) using the product rule again.
            \item Substitute \(\phi_n''\) and \(\phi_n\) into the transformed TISE.
            \item Cancel the common factor \(e^{-y^2/2}\) (which is non-zero).
            \item Simplify the resulting algebraic equation involving \(H_n, H_n', H_n''\) to show it is equivalent to the Hermite equation \(H_n'' - 2yH_n' + 2nH_n = 0\).
        \end{itemize}
\end{enumerate}

\paragraph{Part b): Solutions to Hermite's Equation}
\begin{enumerate}
    \item Verify Rodrigues' Formula:
        \begin{itemize}
            \item Follow the hint: Show \(H_0(y)\) satisfies Hermite's equation for \(n=0\).
            \item **Inductive Step Approach:** Assume \(H_n(y)\) given by Rodrigues' formula satisfies the \(n\)-th Hermite equation. Use properties derivable from Rodrigues' formula (like recurrence relations, e.g., \(H'_{n} = 2nH_{n-1}\) and \(H_{n+1}=2yH_n - 2nH_{n-1}\)) to show that \(H_{n+1}(y)\) constructed via Rodrigues' formula satisfies the \((n+1)\)-th Hermite equation. This involves deriving those recurrence relations from Rodrigues' formula first, or taking them as known properties.
            \item **Alternative (Direct Substitution):** Substitute \(H_n(y)\) from Rodrigues' formula directly into the Hermite equation \(H_n'' - 2yH_n' + 2nH_n = 0\). This requires careful computation of derivatives and simplification; it's a standard but potentially lengthy proof.
        \end{itemize}
    \item Calculate First Three Hermite Polynomials:
        \begin{itemize}
            \item Use Rodrigues' formula \(H_n(y) = (-1)^n e^{y^2} \frac{d^n}{dy^n} e^{-y^2}\) to compute \(H_0(y)\), \(H_1(y)\), and \(H_2(y)\) explicitly.
        \end{itemize}
    \item Order of \(H_n(y)\):
        \begin{itemize}
            \item Observe the results for \(H_0, H_1, H_2\).
            \item Argue from Rodrigues' formula that \(\frac{d^n}{dy^n} e^{-y^2}\) is of the form \(P_n(y)e^{-y^2}\) where \(P_n(y)\) is a polynomial of degree \(n\). Multiplying by \(e^{y^2}\) leaves \(H_n(y)\) as a polynomial of degree \(n\). Note the leading term arises from differentiating \(e^{-y^2}\) \(n\) times yielding \( (-2y)^n e^{-y^2} \).
        \end{itemize}
\end{enumerate}

\paragraph{Part c): Orthogonality of Hermite Polynomials}
\begin{enumerate}
    \item State the integral to be proven zero for \(n > m\): \( I_{nm} = \int_{-\infty}^{\infty} dy e^{-y^{2}} H_{n}(y) H_{m}(y) \).
    \item Substitute Rodrigues' formula for \(H_n(y)\): \( I_{nm} = \int_{-\infty}^{\infty} dy e^{-y^2} [(-1)^n e^{y^2} (\partial_y^n e^{-y^2})] H_m(y) \). Simplify by canceling \(e^{-y^2}e^{y^2}\).
    \item Integrate by parts \(n\) times. In each step, differentiate \(H_m(y)\) and integrate the derivative of \(e^{-y^2}\). Let \(u = H_m(y)\) (or its derivatives) and \(dv = (\partial_y^k e^{-y^2}) dy\).
    \item Show that the boundary terms at \(\pm\infty\) vanish at each step because the exponential term \(e^{-y^2}\) decays faster than any polynomial term grows.
    \item After \(n\) integrations, the integral becomes \( I_{nm} = (-1)^{2n} \int_{-\infty}^{\infty} dy e^{-y^2} (\partial_y^n H_m(y)) \).
    \item Use the fact that \(H_m(y)\) is a polynomial of degree \(m\).
    \item Since \(n > m\), the \(n\)-th derivative \(\partial_y^n H_m(y)\) is identically zero.
    \item Conclude that the integral \(I_{nm}\) is zero for \(n > m\). By symmetry, it is also zero for \(m > n\), thus proving orthogonality for \(n \neq m\).
\end{enumerate}


\end{document}
