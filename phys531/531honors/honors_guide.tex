\documentclass[11pt]{article}
\usepackage{amsmath, amssymb} % Standard math packages
\usepackage{graphicx}        % For potential figures later
\usepackage{bm}              % For bold math symbols (vectors)
\usepackage{braket}          % For Dirac notation < | >
\usepackage[left=0.4in,right=2in]{geometry} % Set margins

% Custom commands (optional)
\newcommand{\pd}{\partial}
\newcommand{\Tr}{\text{Tr}}
\newcommand{\sgn}{\text{sgn}}
\newcommand{\sigmavec}{\bm{\sigma}} % Bold sigma for vector of Pauli matrices
\newcommand{\Hc}{\mathcal{H}}       % Hamiltonian symbol

\title{Chapter X: Dirac Fermions in Condensed Matter}
\author{A Concise Introduction for Harry}
\date{\today}

\begin{document}
\maketitle

\section{Introduction: Dirac Physics in Solids}

The Dirac equation, fundamental to relativistic quantum mechanics, describes spin-1/2 particles and predicts the energy-momentum relation:
\begin{equation}
E^2 = (|\mathbf{p}|c)^2 + (m_0 c^2)^2.
\label{eq:relativistic_energy}
\end{equation}
Remarkably, its mathematical structure also governs the behavior of electron quasiparticles in certain solids, known as \textbf{Dirac materials}. In these systems, the speed of light $c$ is replaced by a Fermi velocity $v \ll c$, and the rest mass $m_0$ by an effective mass $m$. These quasiparticles often possess a \emph{pseudospin} degree of freedom alongside their momentum. This chapter introduces the Hamiltonians describing these quasiparticles and explores their key properties relevant to your project. We set $\hbar = 1$.

\section{Dirac Hamiltonians in Low Dimensions}

Dirac Hamiltonians are characterized by being linear in momentum. They act on multi-component spinors.

\subsection{One Dimension (1D)}
A typical 1D Dirac Hamiltonian is:
\begin{equation}
\Hc_{1D} = v \hat{p}_x \sigma_x + m \sigma_z.
\label{eq:H_1D}
\end{equation}
Here, $\hat{p}_x = -i \pd/\pd x$, and $\sigma_{x,z}$ are Pauli matrices acting on a 2-component spinor. In momentum space ($\hat{p}_x \to p$), the Hamiltonian becomes $\Hc_{1D}(p) = \begin{pmatrix} m & v p \\ v p & -m \end{pmatrix}$. Its eigenvalues yield the dispersion relation:
\begin{equation}
E_{\pm}(p) = \pm \sqrt{m^2 + (v p)^2}.
\label{eq:dispersion_1D}
\end{equation}
This shows two energy bands separated by a gap $2|m|$ at $p=0$. If $m=0$, the gap closes, forming a \textbf{Dirac point}. (\textit{Project: Find eigenstates and plot dispersion}).

\subsection{Two Dimensions (2D)}
In 2D, relevant for graphene or topological insulator surfaces, a common form is:
\begin{equation}
\Hc_{2D} = v (\hat{p}_x \sigma_x + \hat{p}_y \sigma_y) + m \sigma_z.
\label{eq:H_2D}
\end{equation}
In momentum space ($\hat{\mathbf{p}} \to \mathbf{p}=(p_x, p_y)$), the Hamiltonian matrix is $\Hc_{2D}(\mathbf{p}) = \begin{pmatrix} m & v(p_x - i p_y) \\ v(p_x + i p_y) & -m \end{pmatrix}$. The eigenvalues give a similar dispersion:
\begin{equation}
E_{\pm}(\mathbf{p}) = \pm \sqrt{m^2 + v^2 |\mathbf{p}|^2}.
\label{eq:dispersion_2D}
\end{equation}
Again, a gap $2|m|$ exists at $\mathbf{p}=0$ unless $m=0$, which leads to \textbf{Dirac cones}. (\textit{Project: Find eigenstates and plot dispersion}).

\subsection{Three Dimensions (3D)}
In 3D, the Hamiltonian typically involves $4 \times 4$ matrices ($\alpha_i, \beta$) satisfying the Clifford algebra:
\begin{equation}
\Hc_{3D} = v \sum_{i=x,y,z} \hat{p}_i \alpha_i + m \beta.
\label{eq:H_3D_standard}
\end{equation}
This yields the dispersion $E_{\pm}(\mathbf{p}) = \pm \sqrt{m^2 + v^2 |\mathbf{p}|^2}$. (\textit{Project: Clarify the specific 3D Hamiltonian form given and find its eigenstates/values}).

\section{Momentum Space Geometry: Berry Phase Concepts}

Eigenstates $|\psi(\mathbf{p})\rangle$ contain geometric information revealed as $\mathbf{p}$ varies. The \textbf{Berry connection} measures the infinitesimal phase shift:
\begin{equation}
\mathcal{A}_i(\mathbf{p}) = i \braket{\psi(\mathbf{p}) | \frac{\pd}{\pd p_i} | \psi(\mathbf{p})}.
\label{eq:berry_connection}
\end{equation}
It acts like a vector potential in momentum space. Its curl gives the \textbf{Berry curvature}. In 2D:
\begin{equation}
\Omega(\mathbf{p}) = \frac{\pd \mathcal{A}_y}{\pd p_x} - \frac{\pd \mathcal{A}_x}{\pd p_y}.
\label{eq:berry_curvature_2D}
\end{equation}
$\Omega(\mathbf{p})$ acts like a momentum-space magnetic field. Its integral over the 2D Brillouin zone, for a gapped band, gives a topological invariant, the integer \textbf{Chern number} $C$:
\begin{equation}
C = \frac{1}{2\pi} \int d^2p \, \Omega(\mathbf{p}).
\label{eq:chern_number}
\end{equation}
A non-zero Chern number signals non-trivial topology. (\textit{Project: Calculate $\mathcal{A}_i$ and $\Omega$ for $\Hc_{2D}$}).

\section{Bulk-Boundary Correspondence}

A profound principle connects the bulk topology (like $C$) to the existence of protected boundary states.
\begin{itemize}
    \item \textbf{1D:} A domain wall where the mass $m(x)$ changes sign in $\Hc_{1D}$ binds a zero-energy state (Jackiw-Rebbi mechanism), potentially carrying fractional fermion number. (\textit{Project: Yufei's task}).
    \item \textbf{2D:} A non-zero Chern number $C$ implies $|C|$ gapless, chiral \textbf{edge states} propagating along the boundary. These are robust against disorder and underpin the Quantum Hall Effect (QHE). (\textit{Project: Yufei's task}).
\end{itemize}

\section{Landau Levels for 2D Dirac Fermions}

Applying a perpendicular magnetic field $\mathbf{B} = B \hat{\mathbf{z}}$ to the 2D system \eqref{eq:H_2D} leads to quantization into \textbf{Landau Levels (LLs)}. We replace $\hat{\mathbf{p}}$ with the canonical momentum $\hat{\mathbf{\Pi}} = \hat{\mathbf{p}} - q\mathbf{A}$, where $q$ is the charge (e.g., $q=-e$ for electrons) and $\mathbf{B} = \nabla \times \mathbf{A}$. The components satisfy $[\hat{\Pi}_x, \hat{\Pi}_y] = i q B$.

Following your professor's notes (Section A.3), ladder operators $b, b^\dagger$ are defined based on $\hat{\Pi}_{\pm} = \hat{\Pi}_x \pm i \hat{\Pi}_y$, satisfying $[b, b^\dagger]=1$. The definitions depend on the sign of $qB$. Let $M = mv_0^2$ (using $v_0$ for velocity as in notes) be the mass term and $\Omega_c = \sqrt{2|qB|}v_0$. The notes provide the Hamiltonian in terms of these operators.

\textbf{Case 1: $qB < 0$} (e.g., electrons, $B>0$)
\begin{equation}
\Hc = \begin{pmatrix} M & -\Omega_{c} b \\ -\Omega_{c} b^{\dagger} & -M \end{pmatrix}.
\label{eq:H_LL_qBneg_notes}
\end{equation}

\textbf{Case 2: $qB > 0$} (e.g., electrons, $B<0$)
\begin{equation}
\Hc = \begin{pmatrix} M & -\Omega_{c} b^{\dagger} \\ -\Omega_{c} b & -M \end{pmatrix}.
\label{eq:H_LL_qBpos_notes}
\end{equation}
\textit{Note:} These specific forms might differ slightly from standard textbook derivations; we proceed assuming these are correct for the intended system.

The resulting energy spectrum is indexed by $n \in \mathbb{Z}$:
\begin{equation}
E_{n} = \sgn(n) \sqrt{M^2 + \Omega_c^2 |n|} \quad \text{for } n \neq 0,
\label{eq:LL_Dirac_n_nonzero_notes}
\end{equation}
and a unique zeroth level:
\begin{equation}
E_{0} = \sgn(qB) M.
\label{eq:LL_Dirac_n0_notes}
\end{equation}
This spectrum features $\pm E$ symmetry and a characteristic $\sqrt{|n|B}$ dependence. The $n=0$ level's energy depends crucially on the signs of $qB$ and $M$.

Each level $n$ has a degeneracy $N_{deg} = |qB|A / (2\pi) = BA/\Phi_0$, where $\Phi_0 = 2\pi/|q|$ is the flux quantum (with $\hbar=1$). The eigenstates $|n,k\rangle_D$ (given in the notes) involve specific combinations of the standard oscillator states $| |n| \rangle$ and $| |n|-1 \rangle$. The $n=0$ eigenstate is particularly simple: it's fully polarized in the pseudospin basis, aligned with $\sigma_z$ if $qB>0$ ($E_0=M$) and anti-aligned if $qB<0$ ($E_0=-M$).

The notes also provide the average particle density at zero chemical potential (half-filling), obtained by summing over negative energy states:
\begin{equation}
\bar{n}_0 = -\sgn(qBm) \frac{1}{2} \frac{|qB|}{2\pi} = -\sgn(qBm) \frac{1}{2} \frac{B}{\Phi_0}.
\label{eq:density_half_filling_notes}
\end{equation}
The factor of $1/2$ and the sign dependence arise from the unique nature of the $n=0$ LL relative to the Fermi level. (\textit{Project: Solve for the spectrum (confirming \eqref{eq:LL_Dirac_n_nonzero_notes}, \eqref{eq:LL_Dirac_n0_notes}), calculate degeneracy, discuss spectrum vs $M$, calculate $\bar{n}_0$ and $d\bar{n}_0/dB$}).

\section{Outlook}

The concepts covered – Dirac Hamiltonians, Berry curvature, bulk-boundary correspondence, and Landau levels – are essential tools for understanding modern condensed matter phenomena like the QHE in graphene, topological insulators, Chern insulators, and Weyl semimetals. This project provides practical experience with these foundational calculations.

\end{document}
